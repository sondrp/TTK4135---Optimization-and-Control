\documentclass{article}

\usepackage{graphicx}
\usepackage{amsmath}
\usepackage{amssymb}
\usepackage{amsthm}
\usepackage{listings}
\usepackage{xcolor}
\usepackage{tcolorbox}
\usepackage{caption}
\usepackage[hidelinks]{hyperref}
\setlength{\parindent}{0pt}
\usepackage[margin=0.5in]{geometry}
\usepackage{tikz}
\usepackage{pgfplots}
\pgfplotsset{compat=1.15}
\usepackage{mathrsfs}
\usetikzlibrary{arrows}
\usepackage{import}
\usepackage{float}
\usepackage{transparent}




% \usepackage{pstricks}      % does not work :(
% \usepackage{auto-pst-pdf}

\newcommand{\incfig}[2]{%
  \def\svgwidth{#2\columnwidth}%
  \raisebox{-\height}{\import{./figures/}{#1.pdf_tex}}%
}

\lstset{
  language=Matlab,
  basicstyle=\ttfamily\small,
  numbers=left,
  numberstyle=\tiny\color{gray},
  stepnumber=1,
  numbersep=5pt,
  backgroundcolor=\color[HTML]{e2e8f0},
  showspaces=false,
  showstringspaces=false,
  showtabs=false,
  frame=single,
  rulecolor=\color{black},
  captionpos=b,
  breaklines=true,
  breakatwhitespace=false,
  escapeinside={\%*}{*)},
  commentstyle=\color{green},
}


\newcommand{\Exercise}[1]{
  \begin{titlepage}
    \vbox{ }
    \vbox{ }
    \begin{center}
      \includegraphics[width=0.40\textwidth]{../img/NTNU_logo.png}\\[1cm]
      \textsc{\Large TTK4135 - Optimization and Control}\\[0.5cm]
      \vbox{ }
      { \huge \bfseries Exercise \##1}\\[0.4cm]

      \large
      \emph{Author:}\\
      Sondre Pedersen
      \vfill

      {\large\today}
    \end{center}
  \end{titlepage}
}

\newcommand{\Lecture}[2]{
  \title{\textbf{#1}}
  \date{#2}
}

\definecolor{framecolor}{HTML}{94a3b8}
\definecolor{bgcolor}{HTML}{cbd5e1}

\newtcolorbox{problem}[2][]{
  colback=bgcolor,
  colframe=framecolor,
  fonttitle=\bfseries\color{black},
  title=Problem #2,
  width=\textwidth,
  #1
}

% proof 
\definecolor{proofcolor}{HTML}{10b981}
\definecolor{proofbgcolor}{HTML}{d1fae5}

\newtcolorbox{proofbox}[2][]{
  colback=proofbgcolor,
  colframe=proofcolor,
  fonttitle=\bfseries\color{black},
  title=#2,
  width=\textwidth,
  sharp corners,
  #1
}

\newtheorem{theorem}{Theorem}

\newcommand{\thrm}[1]{
  \begin{center}
    \begin{minipage}[t]{0.6\textwidth}
      \begin{theorem}
        #1
      \end{theorem}
    \end{minipage}
  \end{center}
}


% Subtask 
\newcommand{\SUBTASK}[2]{\medskip\medskip\textbf{\large (#1) #2}\medskip}

% Matrix shorthand
\newcommand{\M}[1]{\begin{bmatrix} #1 \end{bmatrix}}

% Subheader for smaller sections
\newcommand{\subheader}[1]{\vspace{1em}\noindent\underline{\textbf{\large #1}}\vspace{0.5em}}


\lstset{basicstyle=\ttfamily, escapeinside={(*@}{@*)}}

\Lecture{Lecture 13: Unconstrained Optimization}{Feb 20, 2025}

\begin{document}

\maketitle

Now we are going back to the basics. This could have been taught before constrained optimization, but the order
was switched in order to be prepared for the labs in time.

\paragraph{Outline}

\begin{itemize}
  \item Optimality conditions for unconstrained optimization.
  \item Ingredients in gradient descent algorithms for unconstrained optimization
        \begin{itemize}
          \item Descent directions (steepest descent, Newton, Quasi-Newton)
          \item How far to walk in descent direction (line search, trust region)
          \item Termination criteria
        \end{itemize}
  \item Scaling
\end{itemize}

\paragraph{Example use}- Machine Learning

\medskip Unconstrained optimization can be used to learn from data and make predictions. Linear regression for example,
the most basic ML algorithm. Linear least squares has an explicit solution. Nonlinear least squares is a little more complicated. The point is that least squares can be formulated
as an optimization problem. Given a bunch of data, we want to find the best coefficients a, b to get the lowest squared distance to actual data points.

\begin{align*}
  \min_{a, b}\qquad & \sum_{i = 1}^{n}(y_i - ax_i - b)^2
\end{align*}

When the line is linear, we can write the objective function as
\begin{align*}
  \min_{\theta}\qquad & (y-x\theta)^{\top}(y-x\theta) \\
                      & y = \M{y_1                    \\ y_2 \\ \vdots  \\ y_n},\ x = \M{x_1 \\ x_2 \\ \vdots \\ x_n} ,\ \theta = \M{a \\ b}
\end{align*}

This is nice and easy, so we can find a direct solution: $\theta = (x^{\top} x)^{-1}x^{\top}y$. In machine learning however, the functions are not linear, so more powerful methods are required.

\section{Unconstrained Optimization}

Before we had constrained optimization

\begin{align*}
  \min_{x \in \mathbb{R}^{n}} f(x) \text{ s.t. } \left\{
  \begin{aligned}
    c_i(x) = 0,\ i \in \mathcal{E} \\
    c_i(x) \geq 0,\  i  \in\mathcal{I}
  \end{aligned}
  \right.
\end{align*}

Now we take away all the constraints, and only look at $\min_{x \in \mathbb{R}^{n}} f(x)$. Note: from now on me assume is $f(x)$ is `smooth': $f \in C^{1}$ or $f \in \mathcal{C}^{2}$. That is
$\nabla f$ exists and is continuous and $\nabla f^2$ exists and is continuous.

What is a solution? Same as before. $x^*$ is a global solution : $f(x^*) \leq f(x),\  \forall x$.
The local solution is also the same as before. $x^*$ is a local solution: $f(x^*) \leq f(x),\  \forall x  \in\mathcal{N}(x)$.

\paragraph{Necessary condition for optimality}- $x^*$ is a local solution and $f \in C^{1} \implies \nabla f(x^*) = 0$.

\paragraph{Taylor expansions}

The form familiar from calculus:
\[
  f(x) = f(a) + (x - a)f'(x) + \frac{1}{2}(x - a)^2 f''(a) + \dots
  .\]

More useful in this course:
\[
  f(x + h) = f(x) + hf'(x) + \frac{1}{2}h^2 f''(x) + \dots
  .\]

These are equivalent by change of variables. Taylor's theorem is that if f is continuously differentiable, then
\[
  f(x+p) = f(x) + \nabla_{f}(x+tp)^{\top}p,\qquad\text{for some }t \in (0, 1)
  .\]

For second order (f is twice continuously differentiable):
\[
  f(x+p) = f(x) + \nabla f(x)^{\top} p + \frac{1}{2}p^{\top} \nabla ^2 f(x+tp)p,\  \qquad\text{for some } t  \in (0, 1)
.\]

\paragraph{Sufficient conditions for optimality}- $\nabla f(x^*) = 0$ and $\nabla ^2 f(x^*) > 0 \implies x^*$ strict local solution. 

\section{General algorithm}

For solving $\min_{x} f(x)$.

\medskip
\lstset{basicstyle=\ttfamily}
\begin{lstlisting}
1. Initial guess (*@$x_0$@*), k = 0
2. While termination criteria not fulfilled:
3.  Find descent direction (*@$p_k$@*) (in (*@$x_k$@*))
4.  Step along (*@$p_k$@*) to (*@$x_{k+1}$@*): (*@$x_{k+1}$@*) = (*@$x_k + \alpha_k p_k$@*)
5.  k = k + 1
6. end  
\end{lstlisting}

\paragraph{Termination criteria}- Given a small tolerance $\epsilon > 0$. Here is a list of different criteria:
\begin{itemize}
  \item $||x_k - x^*|| \leq \epsilon$ or $|f(x_k) - f(x^*) | \leq \epsilon$ (not possible, we don't know $x^*$)
  \item $||\nabla f(x_k)|| \leq \epsilon$
  \item $||x_k - x_{k-1}|| \leq \epsilon$
  \item $|f(x_n) - f(x_{n-1})| \leq \epsilon$
  \item $k > k_{max}$
\end{itemize}

\paragraph{Descent directions}- There are different ways to pick a direction to move:
\begin{itemize}
  \item Steepest descent: $p = - \nabla f(x_k)$
  \item Newton: approximate $f(x)$ around $x_k$. $f(x_k+p) \approx + \nabla f(x_k)^{\top}p + \frac{1}{2}p^{\top}\nabla ^2 f(x_k)p := m_k(p)$.
  \item Quasi-Newton: the same as Newton, but use an approximation for the Hessian. 
\end{itemize} 

We call the function $m_k(p)$ the model function. We want to find the p that minimize the model function:
\begin{align*}
  \min_p \qquad& m_k(p) \rightarrow \nabla_{p} m_k(p) = 0  \\ 
  & \nabla_p m_k(p) = \nabla f(x_k) + \nabla ^2 f (x_k) p = 0  \\ 
  \rightarrow & p = -[\nabla ^2f(x_k)]^{-1} \nabla f(x_k)    
\end{align*}

How does the Newton direction differ from the steepest descent? When the function is quadratic, the Newton direction goes directly towards the local optimum for a quadratic approximation. T

\paragraph{Step length}- How far should we walk along $p_k$? This is also known as globalization strategies - making things work when we are far from the optimum. 

\begin{itemize}
  \item Line search: Find $\alpha$ that approximately solve $\min_{\alpha} f(x_k + \alpha p_n) \rightarrow \alpha_k$
  \item Trust region (not covered in this course)
\end{itemize}

It's hard to say which one of these methods is the best. It depends on the problem. 



\end{document}
