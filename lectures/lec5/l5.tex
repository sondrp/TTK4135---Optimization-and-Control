\documentclass{article}

\usepackage{graphicx}
\usepackage{amsmath}
\usepackage{amssymb}
\usepackage{amsthm}
\usepackage{listings}
\usepackage{xcolor}
\usepackage{tcolorbox}
\usepackage{caption}
\usepackage[hidelinks]{hyperref}
\setlength{\parindent}{0pt}
\usepackage[margin=0.5in]{geometry}
\usepackage{tikz}
\usepackage{pgfplots}
\pgfplotsset{compat=1.15}
\usepackage{mathrsfs}
\usetikzlibrary{arrows}
\usepackage{import}
\usepackage{float}
\usepackage{transparent}




% \usepackage{pstricks}      % does not work :(
% \usepackage{auto-pst-pdf}

\newcommand{\incfig}[2]{%
  \def\svgwidth{#2\columnwidth}%
  \raisebox{-\height}{\import{./figures/}{#1.pdf_tex}}%
}

\lstset{
  language=Matlab,
  basicstyle=\ttfamily\small,
  numbers=left,
  numberstyle=\tiny\color{gray},
  stepnumber=1,
  numbersep=5pt,
  backgroundcolor=\color[HTML]{e2e8f0},
  showspaces=false,
  showstringspaces=false,
  showtabs=false,
  frame=single,
  rulecolor=\color{black},
  captionpos=b,
  breaklines=true,
  breakatwhitespace=false,
  escapeinside={\%*}{*)},
  commentstyle=\color{green},
}


\newcommand{\Exercise}[1]{
  \begin{titlepage}
    \vbox{ }
    \vbox{ }
    \begin{center}
      \includegraphics[width=0.40\textwidth]{../img/NTNU_logo.png}\\[1cm]
      \textsc{\Large TTK4135 - Optimization and Control}\\[0.5cm]
      \vbox{ }
      { \huge \bfseries Exercise \##1}\\[0.4cm]

      \large
      \emph{Author:}\\
      Sondre Pedersen
      \vfill

      {\large\today}
    \end{center}
  \end{titlepage}
}

\newcommand{\Lecture}[2]{
  \title{\textbf{#1}}
  \date{#2}
}

\definecolor{framecolor}{HTML}{94a3b8}
\definecolor{bgcolor}{HTML}{cbd5e1}

\newtcolorbox{problem}[2][]{
  colback=bgcolor,
  colframe=framecolor,
  fonttitle=\bfseries\color{black},
  title=Problem #2,
  width=\textwidth,
  #1
}

% proof 
\definecolor{proofcolor}{HTML}{10b981}
\definecolor{proofbgcolor}{HTML}{d1fae5}

\newtcolorbox{proofbox}[2][]{
  colback=proofbgcolor,
  colframe=proofcolor,
  fonttitle=\bfseries\color{black},
  title=#2,
  width=\textwidth,
  sharp corners,
  #1
}

\newtheorem{theorem}{Theorem}

\newcommand{\thrm}[1]{
  \begin{center}
    \begin{minipage}[t]{0.6\textwidth}
      \begin{theorem}
        #1
      \end{theorem}
    \end{minipage}
  \end{center}
}


% Subtask 
\newcommand{\SUBTASK}[2]{\medskip\medskip\textbf{\large (#1) #2}\medskip}

% Matrix shorthand
\newcommand{\M}[1]{\begin{bmatrix} #1 \end{bmatrix}}

% Subheader for smaller sections
\newcommand{\subheader}[1]{\vspace{1em}\noindent\underline{\textbf{\large #1}}\vspace{0.5em}}


\Lecture{Lecture 5: Solving LPs - the Simplex method}{January 23, 2025}

\begin{document}

\maketitle

\section{Recap from last time}

General linear programming problem looks like
\[
  \min_{x\in \mathbb{R}^{n}} c^{\top}x \qquad \text{subject to}  \qquad \left\{
  \begin{aligned}
    a_i^{\top}x & = b_i,\ i\in \mathcal{E}    \\
    a_i^{\top}x & \geq b_i,\ i\in \mathcal{I}
  \end{aligned}
  \right.
  .\]

These can always be reformulated into standard form
\[
  \min_{x\in \mathbb{R}^{n}}c^{\top}x \qquad \text{subject to} \qquad \left\{
  \begin{aligned}
    Ax & = b    \\
    x  & \geq 0
  \end{aligned}
  \right. \qquad \text{ where } A\in\mathbb{R}^{m \times n}, m < n
  .\]

The lagrangian is now
\[
  \mathcal{L}(x, \lambda, s) = c^{\top}x - \lambda^{\top}(Ax - b) - s^{\top}x
  .\]

The KKT conditions for linear programming:
\begin{align*}
  A^{\top}\lambda^* + s^* & = c                         \\
  Ax^*                    & = b                         \\
  x^*                     & \geq 0                      \\
  s^*                     & \geq 0                      \\
  x_i^*s_i ^*             & = 0 \qquad i = 1,2 ,\dots,n
\end{align*}

\paragraph{Duality}- Primal and dual have the same KKT conditions.

\medskip
\begin{minipage}[c]{0.5\textwidth}
  \centering
  Primal problem
  \begin{align*}
    \min_{x} \quad    & c^{\top}x \\
    \text{s.t.} \quad & Ax = b    \\
                      & x \geq 0  \\
  \end{align*}
\end{minipage}
\begin{minipage}[c]{0.2\textwidth}
  \centering
  Dual problem
  \begin{align*}
    \max_{\lambda, s} \quad & b^{\top}\lambda         \\
    \text{s.t.} \quad       & A^{\top}\lambda + s = c \\
                            & s \geq 0
  \end{align*}
\end{minipage}

\section{Basic feasible points}

A point x is a basic feasible point if
\begin{itemize}
  \item x is feasible
  \item There is an index set $\mathcal{B}(x) \subset \{1,2,,\dots,n\}$ such that
        \begin{itemize}
          \item $\mathcal{B}(x)$ contains m indices
          \item $i \notin \mathcal{B}(x) \implies x_i = 0$
          \item $B = \M{A_i}_{i\in \mathcal{B}(x)}$ is non-singular, $B\in\mathbb{R}^{m \times m}$
        \end{itemize}
  \item $\mathcal{B}(x)$ is called a basis for the LP
  \item The indices not in $\mathcal{B}(x)$ are called $\mathcal{N}(x)$
\end{itemize}

These terms are hard to get a grasp on, so here is an example.

\begin{center}
  \incfig{point}{0.5}
\end{center}

If $\mathcal{B}(x) = 2$ (the red point). We see that $x_1$ and $x_3$ is 0 at this point, but $x_2$ is not.
$A = \M{A_1 & A_2 & A_3}\in \mathbb{R}^{1\times 3}$ and $B = A_2$, a scalar in this case.

\medskip For a different example, with $n = 5$ (five variables) and $m = 2$ (two constraints). Say $\mathcal{B}(x) = \{2, 3\}$
Now
\begin{align*}
  Ax = b   &                               \\
  \M{\cdot & \cdot & \cdot & \cdot & \cdot \\ \cdot & \cdot & \cdot & \cdot & \cdot}& \M{0 \\ \cdot \\ \cdot \\ 0 \\ 0} = \M{0 \\ 0}
\end{align*}

The dots indicate a non-zero value. The second and third column in A make up B, $B = \M{A_i}_{i\in \mathcal{B}(x)}$

\section{Simplex method}

\thrm{(Fundamental theorem of Linear Programming) For standard form LP
  \begin{itemize}
    \item If there is a feasible point, there is a BFP
    \item If the LP has a solution, one solution is a BOP
    \item If the LP is feasible and bounded, there is a solution
  \end{itemize}}

\thrm{All vertices of the feasible polytope
  \[
    \{x | Ax = b,\ x \geq\}
    .\]
  are BFPs (and all BFPs are vertices)
}

One problem that can occur - Degeneracy. This is when a BFP x with $x_i = 0$ for some $i\in \mathcal{B}(x)$
\thrm{If an LP is bounded and non-degenerate, the Simplex method terminates at a BOP}

But in practice, and with a good algorithm, this is not a big problem.
\paragraph{LP KKT conditions} (necessary \& sufficient)

\medskip The Simplex method iterates BFPs until one that fulfills KKT is found. Each step is a move from a vertex to a neighboring vertex (one change in the basis),
that decreases the objective. Another example
\[
  \M{A_{11} & A_{12} & A_{13} & A_{14} & A_{15} \\ A_{21} & A_{22} & A_{23} & A_{24} & A_{25}} \M{0 \\ x_2 \\ x_3 \\ 0 \\ 0} = \M{b_1 \\ b_2}
  .\]

x is a BFP:
\begin{align*}
  \mathcal{B}(x) &= \dots & \mathcal{N}(x) &= \{1, \dots, n\} \backslash \mathcal{B}(x) \\
  B &= \M{A_i}_{i \in \mathcal{B}(x)} & N &= \M{A_i}_{i \in \mathcal{N}(x)} = \M{A_{11} & A_{14} & A_{15} \\ A_{21} & A_{24} & A_{25} } \\
  x_B &= \M{x_i}_{i\in \mathcal{B}(x)} & x_N &= \M{x_i}_{i\in \mathcal{N}(x)} = \M{0 & 0 & 0} ^{\top} \\
  s_B &= \M{s_i}_{i\in \mathcal{B}(x)} & s_N &= \M{s_i}_{i\in \mathcal{N}(x)} \\
  c_B &= \M{c_i}_{i\in \mathcal{B}(x)} & c_N &= \M{c_i}_{i\in \mathcal{N}(x)}
\end{align*}

\paragraph{One step of Simplex-algorithm}- Given a BFP x and $B(x)$
\begin{itemize}
  \item KKT-2: $Ax = Bx_B + Nx_N = Bx_B = b$
  \item KKT-3: $x_B B^{-1}b \geq 0,\  x_N = 0$
  \item KKT-5: $x^{\top}s = x_B^{\top}s_b + x_N^{\top}s_n = x_b^{\top}s_B$, set $s_B = 0$
  \item KKT-1: $\M{s_B \\ s_N} + \M{B^{\top} \\ N^{\top}}\lambda = \M{c_B \\ c_N} \implies \left\{
    \begin{aligned}
      \lambda &= (B^{\top})^{-1}c_B \\
      s_N &= c_N - N^{\top}\lambda
    \end{aligned}
  \right.$
  \item KKT-4: Ok if $s_N  \geq 0 \implies$ BFP is BOP, so we are done
\end{itemize}

If $s_N \ngeq 0$: find a new BFP. (i) choose one index $q\in \mathcal{N}$ s.t. $s_q < 0$, and (ii) increase $x_q$ along $Ax = b$ until a component becomes zero. 

$Ax^{+} = Bx_B^{+} = b = Bx_B$

\medskip $x_B^{+} = x_B - B^{-1}Aqx_q^{+}$, here $d=B^{-1}Aq$

\medskip $x_{B,i}-d_i x_{q,i}^{\top} = 0$

\medskip $x_q^{+} = \min_{i,d_i \geq 0} \{ \frac{x_{B,i}}{d_i}\}$

\begin{center}
  \incfig{path}{0.6}
\end{center}

\paragraph{Ex. 13.1}- 

\medskip Given:
\begin{align*}
  \min_{x \in \mathbb{R}^{2}} \qquad & -4x_1 - 2x_2 \\
  \text{s.t.}     \qquad           & x_1 + x_2 \leq 5 \\
  & 2x_1 + \frac{1}{2} \leq 8 \\
  & x_1 \geq 0,\ x_2 \geq 0
\end{align*}

First write it on standard form:
\begin{align*}
  \min_{x\in\mathbb{R}^{4}} \qquad &-4x_2-2x_2 \\
  \text{s.t.} \qquad & x_1 + x_2 + x_3 = 5 \\
  & 2x_1 + \frac{1}{2} x_2+ x_4 = 8 \\
  & x \geq 0
\end{align*}

On matrix form: 
\[
  c = \M{-4 \\ -2  \\ 0 \\ 0},\ A = \M{1 & 1 & 1 & 0 \\ 2 & \frac{1}{2} & 0 & 1},\ b = \M{5  \\ 8}
.\] 

Choose first basic feasible point $x = \M{0 \\ 0 \\ 5 \\ 8}$. We can directly read out these values from the problem:

\begin{itemize}
  \item $x = \M{0  & 0 & 5 & 8}^{\top}$
  \item $\mathcal{B}(x) = \{3, 4\}$
  \item $\mathcal{N}(x) = \{1,2\}$
  \item $B = \M{1  & 0 \\ 0 & 1}$
  \item $N = \M{1 & 1 \\ 2 &  \frac{1}{2}}$
\end{itemize}

\medskip Calculate some values (?):
\begin{itemize}
  \item $x_B = B^{-1}b = \M{5  \\ 8}$
  \item $\lambda = (B^{\top})^{-1}c_B = \M{1  & 0 \\ 0 & 1}\M{0 \\ 0} = \M{0 \\ 0}$
  \item $s_N = c_N - N^{\top}\lambda = \M{-4 \\ -2}-\M{1 & 2 \\ 1 &  \frac{1}{2}}\M{0 \\ 0} = \M{-4 \\ -2} \qquad \ngeq 0$
\end{itemize}

Pick smallest element of $s_N$: $s_1=-4 \rightarrow q = 1$. Let $q = 1$ enter $\mathcal{B}$ (leaving $\mathcal{N}$). Increase $x_1$ while $Ax=b,\  x_4$ becomes zero first. Let $p=4$ leave $\mathcal{B}$ and enter $\mathcal{N}$.

\medskip \textbf{Second iteration}
\begin{itemize}
  \item $\mathcal{B}(x) = \{1, 3\}$
  \item $\mathcal{N}(x) = \{2, 4\}$
  \item $B = \M{1 & 1 \\ 2 & 0} \rightarrow B^{-1} = \frac{1}{2}\M{0 & 1 \\ 2 & -1}$
  \item $N = \M{1 & 0 \\  \frac{1}{2} & 1}$
\end{itemize}

Calculations: 
\begin{itemize}
  \item $x_B = \M{x_1  \\ x_3} = B^{-1}b = \frac{1}{2} \M{0 & 1 \\ 2 & -1}\M{5 \\ 8} = \M{4 \\ 1}$
  \item $x_N = \M{0  \\ 0}$
  \item $\lambda = (B^{\top})^{-1}c_B = \dots = \M{0 \\ -2}$
  \item $s_N = c_N - N^{\top}\lambda = \M{-2 \\ 0}- \M{1 &  \frac{1}{2} \\ 0 & 1}\M{0 \\ -2} = \M{-1 \\ 2}$
\end{itemize}

Now the smallest value is $q = 2$, the entering index. $p = 3$ is the leaving index. 

\medskip \textbf{Third iteration}
\begin{itemize}
  \item $\mathcal{B}(x) = \{1, 2\}$
  \item $\mathcal{N}(x) = \{3, 4\}$
  \item $B = \M{1 & 1 \\ 2 &  \frac{1}{2}} \rightarrow B^{-1} = \frac{1}{3}\M{-1 & 2 \\ 4 & -2}$
  \item $N = \M{1 & 0 \\ 0 & 1}$
  \item 
\end{itemize}
And
\begin{itemize}
  \item $x_B = B^{-1}b = \dots = \M{ 11 / 3  \\  4 / 3}$
  \item $\lambda = (B^{-1})^{\top}c_B = \dots = \M{ -4 / 3 \\ -4 / 3}$
  \item $s_N = c_N -N^{\top}\lambda = \M{0 \\ 0} - \M{1 & 0 \\ 0 & 1}\M{- 4 / 3  \\  -4 /3} = \M{9 / 3 \\ 4 /3} \geq 0$
\end{itemize}

Since $s_N$ is greater than 0, we have an optimal solution, $x^* = \M{11 /3  \\ 4 /3  \\  0  \\  0}$

\section{Other details}

In each step of the Simplex algorithm, two linear systems must be solved. $B^{\top}\lambda = c_B$ and $Bd = A_q$. The last
one is to find the direction to check when increasing $x_q$. We also have $Bx_B = b$. Since $x_B$ is not needed in the iterations, we don't need
to solve this before the final iteration. 

\medskip B is a general, non-singular matrix. It is guaranteed to have a solution. LU factorization is the correct method to solve both systems.

\medskip In each step of the Simplex method, one column of B is replaced. There are some tricks to save time with maintaining solutions of LU factorizations.  

\paragraph{Finding a starting point} in the Simplex method is as hard as solving the problem itself. 
Normally, simplex algorithms have two phases, one for finding the first feasible point, and the other to actually solve it. 

This is done by making another LP with a trivial BFP, where the solution is a BFP for the original problem. 
\begin{align*}
  &\min e^{\top}z \text{ subject to } Ax + Ez = b,\ (x,z) > 0 \\ 
  &e = (1,1,,\dots,1)^{\top},\  \text{ E diagonal matrix with }\left\{
    \begin{aligned}
      &E_{jj} = 1 \text{ if } b_j \geq 0 \\ 
      &E_{jj} = -1 \text{ if } b_j < 0  
    \end{aligned}
  \right.
\end{align*}

\paragraph{Complexity}- the simplex method is typically very efficient (2m to 3m  iterations). But the worst case is exponential complexity, if all vertices must be visited.

Interior point methods have better worst case complexity, but simplex still performs better usually. 
\end{document}
