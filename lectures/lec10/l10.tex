\documentclass{article}

\usepackage{graphicx}
\usepackage{amsmath}
\usepackage{amssymb}
\usepackage{amsthm}
\usepackage{listings}
\usepackage{xcolor}
\usepackage{tcolorbox}
\usepackage{caption}
\usepackage[hidelinks]{hyperref}
\setlength{\parindent}{0pt}
\usepackage[margin=0.5in]{geometry}
\usepackage{tikz}
\usepackage{pgfplots}
\pgfplotsset{compat=1.15}
\usepackage{mathrsfs}
\usetikzlibrary{arrows}
\usepackage{import}
\usepackage{float}
\usepackage{transparent}




% \usepackage{pstricks}      % does not work :(
% \usepackage{auto-pst-pdf}

\newcommand{\incfig}[2]{%
  \def\svgwidth{#2\columnwidth}%
  \raisebox{-\height}{\import{./figures/}{#1.pdf_tex}}%
}

\lstset{
  language=Matlab,
  basicstyle=\ttfamily\small,
  numbers=left,
  numberstyle=\tiny\color{gray},
  stepnumber=1,
  numbersep=5pt,
  backgroundcolor=\color[HTML]{e2e8f0},
  showspaces=false,
  showstringspaces=false,
  showtabs=false,
  frame=single,
  rulecolor=\color{black},
  captionpos=b,
  breaklines=true,
  breakatwhitespace=false,
  escapeinside={\%*}{*)},
  commentstyle=\color{green},
}


\newcommand{\Exercise}[1]{
  \begin{titlepage}
    \vbox{ }
    \vbox{ }
    \begin{center}
      \includegraphics[width=0.40\textwidth]{../img/NTNU_logo.png}\\[1cm]
      \textsc{\Large TTK4135 - Optimization and Control}\\[0.5cm]
      \vbox{ }
      { \huge \bfseries Exercise \##1}\\[0.4cm]

      \large
      \emph{Author:}\\
      Sondre Pedersen
      \vfill

      {\large\today}
    \end{center}
  \end{titlepage}
}

\newcommand{\Lecture}[2]{
  \title{\textbf{#1}}
  \date{#2}
}

\definecolor{framecolor}{HTML}{94a3b8}
\definecolor{bgcolor}{HTML}{cbd5e1}

\newtcolorbox{problem}[2][]{
  colback=bgcolor,
  colframe=framecolor,
  fonttitle=\bfseries\color{black},
  title=Problem #2,
  width=\textwidth,
  #1
}

% proof 
\definecolor{proofcolor}{HTML}{10b981}
\definecolor{proofbgcolor}{HTML}{d1fae5}

\newtcolorbox{proofbox}[2][]{
  colback=proofbgcolor,
  colframe=proofcolor,
  fonttitle=\bfseries\color{black},
  title=#2,
  width=\textwidth,
  sharp corners,
  #1
}

\newtheorem{theorem}{Theorem}

\newcommand{\thrm}[1]{
  \begin{center}
    \begin{minipage}[t]{0.6\textwidth}
      \begin{theorem}
        #1
      \end{theorem}
    \end{minipage}
  \end{center}
}


% Subtask 
\newcommand{\SUBTASK}[2]{\medskip\medskip\textbf{\large (#1) #2}\medskip}

% Matrix shorthand
\newcommand{\M}[1]{\begin{bmatrix} #1 \end{bmatrix}}

% Subheader for smaller sections
\newcommand{\subheader}[1]{\vspace{1em}\noindent\underline{\textbf{\large #1}}\vspace{0.5em}}


\Lecture{Lecture 10: Model Predictive Control}{February 6, 2025}

\begin{document}

\maketitle

\section{Intro}

Model predictive control started in chemical process control. Before MPC it was common to have multiple single loop
pid controllers. MPC then came to replace these with a single optimization algorithm. But MPC is used in a wide range of 
application. Some of these are time sensitive, and require the optimal solution to be found fast. 

\paragraph{Open-loop optimization with linear state-space model}

\begin{align*}
  \min_{z\in \mathbb{R}^{n}} f(z) &= \sum_{t=0}^{N-1} \frac{1}{2} x_{t+1}^{\top}Q_{t+1}x_{t+1} + d_{x,t+1}x_{t+1} + \frac{1}{2}u_t^{\top}R_t u_t + d_{u,t}u_t + \frac{1}{2} \Delta u_t^{\top}S \Delta u_t \\ 
  \text{subject to}& \\ 
  x_{t+1} &= A_t x_t + B_t u_t  \\ 
  x^{\text{low}} &\leq x_t \leq x^{\text{high}} \\ 
  u^{\text{low}} & \leq u_t \leq u^{\text{high}} \\ 
  -\Delta u^{\text{high}} & \leq \Delta u_t \leq \Delta u^{\text{high}}  \\ 
\end{align*}

The MPC is a QP, which can be solved with techniques from previous lectures. 

\paragraph{Model predicive control principle}- 

\medskip Is that MPC works by measuring the plant state, solve the optimization problem, and then find a sequence of inputs (1st, 2nd, ...) until the end 
of the horizon. Then you use the first of these inputs. 

This might at first seem like a lot of wasted computing time, since we are only using some of the results. The point is that we can now use these results to compare. This introduces feedback. 

\section{The Chess analogy}

\begin{center}
  \incfig{chess}{0.7}
\end{center}

The optimization algorithm will consider many possible futures, and land on the optimal one. To pick
and optimal move, you have to consider a series of moves, and play the first one in sequence. If the opponent plays according to the sequence, 
all is good. But if they should pick another move, the MPC must run the optimization algorithm again. It obviously does not make sense to blindly follow 
the original plan without caring about the opponents moves. 

This might not be the best analogy, but since it is about chess I had to include it. 

\section{The feasibility problem}

Inequality constraints on states may imply that for some $x_0$, there are no solution to the MPC QP. 

\begin{align*}
  \min_z \qquad& \sum_{t=0}^{N-1} \frac{1}{2} x_{t+1}^{\top}Qx_{t+1} + \frac{1}{2} u_t^{\top}Ru_t \\ 
  \text{s.t.}\qquad& x_{t+1} = Ax_t + Bu_t  \\ 
  & x^{\text{low}} \leq t_t \leq x^{\text{high}} \\ 
  & u^{\text{low}} \leq u_t \leq u^{\text{high}}
\end{align*}

How can we change this in order to always have a solution? Introduce `slack' variables ($\epsilon$). These are added to the boundary conditions, and we get

\begin{align*}
  \min_{x, \epsilon} \qquad& \sum_{t=0}^{N-1} \frac{1}{2} x_{t+1}^{\top}Qx_{t+1} + \frac{1}{2} u_t^{\top}Ru_t + \delta ^{\top} \epsilon \\ 
  \text{s.t.}\qquad& x_{t+1} = Ax_t + Bu_t  \\ 
  -\epsilon + & x^{\text{low}} \leq t_t \leq x^{\text{high}} + \epsilon \\ 
  & u^{\text{low}} \leq u_t \leq u^{\text{high}}
\end{align*}

The epsilon should be as small as possible, so a weight for this is added in the objective function. 

\medskip There is a concept called `Exact penalty' which says that $\delta$ is large enough $ \implies \epsilon = 0$ whenever possible. 

\section{Is MPC always stable?}

Design MPC for $x_{t+1} = 1.2x_t + u_t$, no constraint, $N = 2$. The MPC problem looks like

\begin{align*}
  \min_{u_0, u_1, x_1, x_2}\qquad& \sum_{t = 0}^{1} \frac{1}{2} x_{t+1}^2 + \frac{r}{2} u_t^2  \\ 
  \text{s.t.}\qquad& x_1 = 1.2x_0 + u_0 \\ 
  & x_2 = 1.2 x_1 + u_1
\end{align*}

with a tuning varialbe r. This problem is small enough to solve directly. 
\begin{align*}
  x_2 &= 1.2(1.2x_0 + u_0) + u_1 = 1.44x_0 + 1.2u_0 + u_1 \\ 
  &\text{Restate the optimization problem in terms of u only} \\ 
  \min_{x_0, u_1} \qquad & \frac{1}{2}(1.2x_0+u_0)^2 + \frac{1}{2}(1.44x_0 + 1.2u_0 + u_1)^2 + \frac{r}{2}u_0^2 + \frac{r}{2}u_1^2 = f(u_0, u_1)
\end{align*}

It can be shown that f is convex here, so the solution is found by $\frac{\partial f}{\partial u_0} = 0$, $\frac{\partial f}{\partial u_1} = 0$.

In the end we get that $u_0 = - \frac{1.2+2.928r}{1+3.44r+r^2}x_0$. 

\medskip The point is that when r gets to big here, the MPC closed loop becomes unstable. So MPC is not automatically stable. 

\section{MPC and stability}

Nominal vs robust stability
\begin{itemize}
  \item Nominal stability: Stability when optimization model = plant model. No `model-plant mismatch', no disturbances.
  \item Robust stability: Stability when optimization model $\neq$ plant model. `Model-plant mismatch' and/or disturbances
\end{itemize}

Requirements for nominal stability (as for infinite horizon LQ control):
\begin{itemize}
  \item Stabilizability ( (A, B) stabilizable )
  \item Detectability ( (A, D) detectable )
  \begin{itemize}
    \item D is a matrix such that $Q = D^{\top}D$ (that is, `D is matrix square root of Q')
    \item Detectability: No modes can grow to infinity without being `visible' through Q
  \end{itemize}
\end{itemize}

\paragraph{How to achieve nominal stability?}- 

\medskip Three different theoretical solutions: 
\begin{itemize}
  \item Choose prediction horizon equal to infinity (N = $\infty$). This is usually not possible
  \item For given N, design Q and R such that MPC is stable. This is difficult in general. Not recommended.
  \item Change the optimziation problem. Add a terminal cost and terminal constraint to the optimization problem. 
\end{itemize}

In practice, you can choose N that is large enough. Two rules of thumb: (i) choose N longer than the `dominating dynamics' and/or (ii) choose N large enough such that 
open-loop predictions resemble closed-loop. 

\paragraph{Why MPC over PID control?}- 

\medskip MPC handles constraints in a transparent way. Handling the constraints with PID can require a lot of logic, which might be hard to understand. 
MPC is also multivariable. For PID controllers, you might have problems where one input strongly affects other states.  

\end{document}