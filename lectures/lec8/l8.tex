\documentclass{article}

\usepackage{graphicx}
\usepackage{amsmath}
\usepackage{amssymb}
\usepackage{amsthm}
\usepackage{listings}
\usepackage{xcolor}
\usepackage{tcolorbox}
\usepackage{caption}
\usepackage[hidelinks]{hyperref}
\setlength{\parindent}{0pt}
\usepackage[margin=0.5in]{geometry}
\usepackage{tikz}
\usepackage{pgfplots}
\pgfplotsset{compat=1.15}
\usepackage{mathrsfs}
\usetikzlibrary{arrows}
\usepackage{import}
\usepackage{float}
\usepackage{transparent}




% \usepackage{pstricks}      % does not work :(
% \usepackage{auto-pst-pdf}

\newcommand{\incfig}[2]{%
  \def\svgwidth{#2\columnwidth}%
  \raisebox{-\height}{\import{./figures/}{#1.pdf_tex}}%
}

\lstset{
  language=Matlab,
  basicstyle=\ttfamily\small,
  numbers=left,
  numberstyle=\tiny\color{gray},
  stepnumber=1,
  numbersep=5pt,
  backgroundcolor=\color[HTML]{e2e8f0},
  showspaces=false,
  showstringspaces=false,
  showtabs=false,
  frame=single,
  rulecolor=\color{black},
  captionpos=b,
  breaklines=true,
  breakatwhitespace=false,
  escapeinside={\%*}{*)},
  commentstyle=\color{green},
}


\newcommand{\Exercise}[1]{
  \begin{titlepage}
    \vbox{ }
    \vbox{ }
    \begin{center}
      \includegraphics[width=0.40\textwidth]{../img/NTNU_logo.png}\\[1cm]
      \textsc{\Large TTK4135 - Optimization and Control}\\[0.5cm]
      \vbox{ }
      { \huge \bfseries Exercise \##1}\\[0.4cm]

      \large
      \emph{Author:}\\
      Sondre Pedersen
      \vfill

      {\large\today}
    \end{center}
  \end{titlepage}
}

\newcommand{\Lecture}[2]{
  \title{\textbf{#1}}
  \date{#2}
}

\definecolor{framecolor}{HTML}{94a3b8}
\definecolor{bgcolor}{HTML}{cbd5e1}

\newtcolorbox{problem}[2][]{
  colback=bgcolor,
  colframe=framecolor,
  fonttitle=\bfseries\color{black},
  title=Problem #2,
  width=\textwidth,
  #1
}

% proof 
\definecolor{proofcolor}{HTML}{10b981}
\definecolor{proofbgcolor}{HTML}{d1fae5}

\newtcolorbox{proofbox}[2][]{
  colback=proofbgcolor,
  colframe=proofcolor,
  fonttitle=\bfseries\color{black},
  title=#2,
  width=\textwidth,
  sharp corners,
  #1
}

\newtheorem{theorem}{Theorem}

\newcommand{\thrm}[1]{
  \begin{center}
    \begin{minipage}[t]{0.6\textwidth}
      \begin{theorem}
        #1
      \end{theorem}
    \end{minipage}
  \end{center}
}


% Subtask 
\newcommand{\SUBTASK}[2]{\medskip\medskip\textbf{\large (#1) #2}\medskip}

% Matrix shorthand
\newcommand{\M}[1]{\begin{bmatrix} #1 \end{bmatrix}}

% Subheader for smaller sections
\newcommand{\subheader}[1]{\vspace{1em}\noindent\underline{\textbf{\large #1}}\vspace{0.5em}}


\Lecture{Lecture 8: Open-Loop Dynamic Optimization}{January 31, 2025}

\begin{document}

\maketitle

\section{Optimization in Control}

This is the start of the control part of the course.

\begin{center}
  \begin{tabular}{p{6cm}p{10cm}} % Adjust column width as needed
    \textbf{Optimization problems we study} & \textbf{Use of optimization in control}      \\
    \hline
    \begin{itemize}
      \item Class: Linear programming
      \item Method: Simplex
    \end{itemize}
                                            &
    \begin{itemize}
      \item (Some use in MPC, but not typical)
      \item (Sometimes used for computing economic setpoints for PID control)
    \end{itemize}
    \\
    \hline
    \begin{itemize}
      \item Class: Quadratic programming
      \item Method: Active set
    \end{itemize}
                                            & \begin{itemize}
                                                \item Dynamic optimizatin using linear models
                                                      \begin{itemize}
              \item Open loop dynamic optimization
              \item Closed loop dynamic optimization / MPC
              \item Linear Quadratic Control
            \end{itemize}
                                              \end{itemize} \\
    \hline
    \begin{itemize}
      \item Class: Nonlinear programming
      \item Method: Sequential Quadratic Programming
    \end{itemize}
                                            &
    \begin{itemize}
      \item Dynamic optimization using nonlinear models
      \item Nonlinear Model Predictive Control
    \end{itemize}
  \end{tabular}
\end{center}

\paragraph{Static vs dynamic optimization}- Does time matter?

\medskip When solving practical problems (that is, we optimize some process) we have two cases:
\begin{itemize}
  \item Time independent process $\implies$ static optimization
        \begin{itemize}
          \item Common in finance, economic optimization
          \item Farming example from previous lectures
        \end{itemize}
  \item Time dependent process $\implies$ dynamic optimization
        \begin{itemize}
          \item More common in cybernetics/control engineering
          \item Mechanical systems (boat, drone, robot, ...), chemical process (chemical reactor, process plant, ...), electrical/energy system
        \end{itemize}
\end{itemize}

It is possible to go for a third option: quazi-dynamic optimization. If the process changes slowly over time, it is possible to assume that it
is not changing at all and using static optimization. To compensate, re-solve regularly.

\paragraph{Quazi-dynamic example}- Oil production

Given a floating production ship with n wells. These wells will produce a stream of oil, gass and water. The ship has a separator to put these in different tanks.
The ouput of the separator (g: gass, o: oil, w: water) from well i:
\begin{align*}
  q_g & = \sum q_{gi} \\
  q_o & = \sum q_{oi} \\
  q_w & = \sum q_{wi} \\
\end{align*}

From day to day the ouputs do not change very much, so we can make a static model.
\begin{align*}
  (*)             & \left\{
  \begin{aligned}
     & q_{qi} = \alpha_{gi} \times q_i,\  i = 1,\dots,n                     \\
     & q_{wi} = \alpha_{wi} \times q_i,\  i = 1,\dots,n                     \\
     & q_{oi} = (1 - \alpha_{gi} - \alpha_{wi}) \times q_i,\  i = 1,\dots,n \\
     & \sum_{i = 1}^{n} q_{gi} \leq q_{g, max}                              \\
     & \sum_{i = 1}^{n} q_{wi} \leq q_{w, max}                              \\
  \end{aligned}
  \right.                                                                                \\
  \text{Optimize} &                                                                      \\
                  & \min_{z}\quad-\sum_{i = 1}^{n}q_{oi} \qquad\text{s.t.}\qquad (*)     \\
                  & \text{where } z = (q_1, g_{g_1}, q_{wq}, q_{o_1}, q_1, \dots)^{\top}
\end{align*}

Here $\alpha$ is a gass-oil ratio. We assume the parameters are constant for one day, but they might change the next. When they do,
re-optimize.

\section{Dynamic models}
\begin{align*}
  x_{t+1} &= g(x_t, u_t)       && \text{(general nonlinear dynamic model)} \\
  x_{t+1} &= A_t x_t + B_t u_t && \text{(LTV)} \\
  x_{t+1} &= A x_t + Bu_t     && \text{(LTI)}
\end{align*}

In the world, models are not linear. But they can still be a good approximation. To get this approximation, we linearize $x_{t+1} = g(x_t, u_t)$.
Linearize about either $\left\{
  \begin{aligned}
    \text{trajectory }\overline{x}_{t+1} = g(\overline{x}_t, \overline{u}_t) \\
    \text{stationary point } \overline{x} = g(\overline{x}, \overline{u})
  \end{aligned}
\right.$

Define perturbation $x_t = \overline{x}_t + \delta x_t ,\ u_t = \overline{u}_t + \delta u_t$. 
Now 
\begin{align*}
  x_{t+1} &= g(\overline{x}_t+ \delta x_t, \overline{u}_t + \delta u_t) \\
  &\approx g(\overline{x}_t, \overline{u}_t) + \frac{\partial g}{\partial x} \bigg|_{\overline{x}_t, \overline{u}_t}\times \delta x_t + \frac{\partial g}{\partial u}\bigg|_{\overline{x}_t, \overline{u}_t}\times \delta u_t\\
  & \delta x_{t+1} = A_t \delta x_t + B_t \delta u_t  \qquad\text{(LTV)} \\
  & \delta x_{t+1} = A \delta x_t + B \delta u_t  \qquad\text{     (LTI)} \\
\end{align*}

\paragraph{General dynamic optimization problem}- 

\begin{minipage}[t]{0.5\textwidth}
  \begin{align*}
    \min_z \quad &\sum_{t = 0}^{N-1}f_t(x_{t+1}, u_t) \\
    \text{s.t.} \quad &x_{t+1} = g(x_t, u_t),\  t = 0,\dots,N-1 \\ 
    &h_t(x_t, u_t) \leq 0,\  t = 0,\dots,N
  \end{align*}
\end{minipage}
\begin{minipage}[t]{0.5\textwidth}
  \begin{itemize}
    \item N: prediction horizon
    \item $f_t$: stage cost
    \item $u_t$: system input
    \item $g_t$: dynamic model
    \item $h_t$: state and input constraints
  \end{itemize}
\end{minipage}
$x_0$ is given.
$z = \M{u_0^{\top}, x_1^{\top}, u_1^{\top},\dots,u_{N-1}^{\top}, x_N^{\top}}^{\top}$

\begin{center}
  \incfig{system}{1}
\end{center}

Typical objectives (stage costs) in dynamic optimization: penalize deviations from a constant reference/setpoint (regulation), or deviations from a reference trajectory (tracking).
Other typical objectives include: maximize profit, maximize production, minimize costs, limit tear, reach endpoints fast.

\medskip General quadratic stage cost:
\[
  f_t(x_{t+1}, u_t) = \frac{1}{2} x_{t+1}^{\top}Q_t K_{t+1} + d_{x, t}^{\top} x_t + \frac{1}{2} u_t^{\top}R_t u_t + d_{u,t}^{\top} u_t
.\] 

This general form reduces nicely to special cases. 

\medskip A quadratic objective is good for two reasons: (i) it is convenient mathematically. The gradient is linear, and smoothness is good for calculations. 
(ii) It has a natural response to deviations. A small deviation does not matter too much, but big deviations will cause a stronger response. 

\paragraph{Standard linear dynamic optimization problem}
\begin{align*}
  \min_z \qquad& \frac{1}{2} \sum_{t = 0}^{N-1}x_{t+1}^{\top}Qx_{t+1} + \frac{1}{2}u_t^{\top}R u_t \\ 
  \text{s.t.} \qquad& x_{t+1} = Ax_t + Bu_t,\  t = 0,\dots,N-1 \\
  & x^{low}\leq x_t \leq x^{high},\  t = 1,\dots,N \\ 
  & u^{low}\leq u_t \leq u^{high} ,\ t = 0,\dots,N-1 \\\\
  & x_0: \text{ given} \\
  & z = \M{u_0^{\top}, x_1^{\top}, u_1^{\top}, x_2^{\top},\dots,u_{N-1}^{\top}, x_N^{\top}}^{\top}
\end{align*}

Can be solved using `Batch approach v1, "Full space"'. Great name. With this approach we can write the problem as

\begin{align*}
  \min_z \qquad& \frac{1}{2}z^{\top}\M{
    R \\ 
     & Q \\ 
     &  & R \\ 
     &  &  & Q \\ 
     &  &  &  & \ddots
  }z \\
  \text{s.t.}\qquad& \M{
    -B & I & 0 & 0 & 0 & \dots \\ 
    0 & -A & -B & I & 0 & \dots \\ 
    0 & 0 & -A & -B & I & \dots \\ 
    \vdots &  &  & \ddots  \\ 
  }z = \M{
    Ax_0 \\ 
    0 \\  \\ \vdots \\ 
  } \\ 
  & \M{
    I & 0 & 0 & 0 & \dots \\ 
    -I & 0 & 0 & 0 & \dots \\ 
    0 & I & 0 & 0 & \dots \\ 
    0 & -I & 0 & 0 & \dots \\ 
     &  & \vdots
  }z \geq \M{u^{low} \\ -u^{high} \\ x^{low} \\ x^{high} \\ \vdots}
\end{align*}

More common in the industry is `Batch approach v2, "Reduced space"/"condensing"'
\begin{align*}
  x_1 &= Ax_0+Bu_0 \\ 
  x_2 &= Ax_1+Bu_1 = A(Ax_0+Bu_0) + Bu_1 = A^2x_0+ABu_0+Bu_1 \\ 
  x_3 &= Ax_2+Bu_2= A(\dots)+Bu_2 = A^3x_0 + A^2Bu_0 + ABu_1+Bu_2 \\ 
  &\text{Can be written on matrix form} \\ 
  \M{x_1 \\ x_2 \\ x_3 \\ \vdots} &= \M{A \\ A^2 \\ A^3 \\ \vdots}x_0 + \M{B & 0 & 0 & \dots \\ AB & B & 0 & \dots \\ A^2B & AB & B & \dots \\ \vdots & \vdots & \vdots & \ddots}
  \M{u_0 \\ u_1 \\ \vdots \\ u_N}  \\ 
  \rightarrow x &= S^{x}x_0 + S^{u} u
\end{align*}

Use the above formulation to write the dynamic optimization problem as
\begin{align*}
  \min \qquad& \frac{1}{2}(S^{x}x_0+S^{u}u)^{\top}\M{Q \\  & Q \\  &  & \ddots \\  &  &  & Q}(S^{x}x_0 + S^{u}u) + u^{\top}\M{R \\  & R \\  &  & \ddots \\  &  &  & R}u  \\ 
  \text{s.t.}\qquad& \M{x^{low} \\ \vdots \\ x^{low}} \leq S^{x}x_0 + S^{u}u \leq \M{x^{high} \\ \vdots \\ x^{high}} \\ 
  &\M{u^{low} \\ \vdots \\ u^{low}} \leq u \leq \M{u^{high} \\ \vdots  \\ u^{high}}
\end{align*}


\end{document}
