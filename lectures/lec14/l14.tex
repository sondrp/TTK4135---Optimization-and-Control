\documentclass{article}

\usepackage{graphicx}
\usepackage{amsmath}
\usepackage{amssymb}
\usepackage{amsthm}
\usepackage{listings}
\usepackage{xcolor}
\usepackage{tcolorbox}
\usepackage{caption}
\usepackage[hidelinks]{hyperref}
\setlength{\parindent}{0pt}
\usepackage[margin=0.5in]{geometry}
\usepackage{tikz}
\usepackage{pgfplots}
\pgfplotsset{compat=1.15}
\usepackage{mathrsfs}
\usetikzlibrary{arrows}
\usepackage{import}
\usepackage{float}
\usepackage{transparent}




% \usepackage{pstricks}      % does not work :(
% \usepackage{auto-pst-pdf}

\newcommand{\incfig}[2]{%
  \def\svgwidth{#2\columnwidth}%
  \raisebox{-\height}{\import{./figures/}{#1.pdf_tex}}%
}

\lstset{
  language=Matlab,
  basicstyle=\ttfamily\small,
  numbers=left,
  numberstyle=\tiny\color{gray},
  stepnumber=1,
  numbersep=5pt,
  backgroundcolor=\color[HTML]{e2e8f0},
  showspaces=false,
  showstringspaces=false,
  showtabs=false,
  frame=single,
  rulecolor=\color{black},
  captionpos=b,
  breaklines=true,
  breakatwhitespace=false,
  escapeinside={\%*}{*)},
  commentstyle=\color{green},
}


\newcommand{\Exercise}[1]{
  \begin{titlepage}
    \vbox{ }
    \vbox{ }
    \begin{center}
      \includegraphics[width=0.40\textwidth]{../img/NTNU_logo.png}\\[1cm]
      \textsc{\Large TTK4135 - Optimization and Control}\\[0.5cm]
      \vbox{ }
      { \huge \bfseries Exercise \##1}\\[0.4cm]

      \large
      \emph{Author:}\\
      Sondre Pedersen
      \vfill

      {\large\today}
    \end{center}
  \end{titlepage}
}

\newcommand{\Lecture}[2]{
  \title{\textbf{#1}}
  \date{#2}
}

\definecolor{framecolor}{HTML}{94a3b8}
\definecolor{bgcolor}{HTML}{cbd5e1}

\newtcolorbox{problem}[2][]{
  colback=bgcolor,
  colframe=framecolor,
  fonttitle=\bfseries\color{black},
  title=Problem #2,
  width=\textwidth,
  #1
}

% proof 
\definecolor{proofcolor}{HTML}{10b981}
\definecolor{proofbgcolor}{HTML}{d1fae5}

\newtcolorbox{proofbox}[2][]{
  colback=proofbgcolor,
  colframe=proofcolor,
  fonttitle=\bfseries\color{black},
  title=#2,
  width=\textwidth,
  sharp corners,
  #1
}

\newtheorem{theorem}{Theorem}

\newcommand{\thrm}[1]{
  \begin{center}
    \begin{minipage}[t]{0.6\textwidth}
      \begin{theorem}
        #1
      \end{theorem}
    \end{minipage}
  \end{center}
}


% Subtask 
\newcommand{\SUBTASK}[2]{\medskip\medskip\textbf{\large (#1) #2}\medskip}

% Matrix shorthand
\newcommand{\M}[1]{\begin{bmatrix} #1 \end{bmatrix}}

% Subheader for smaller sections
\newcommand{\subheader}[1]{\vspace{1em}\noindent\underline{\textbf{\large #1}}\vspace{0.5em}}


\lstset{basicstyle=\ttfamily, escapeinside={@}{@}}

\Lecture{Lecture 14: Line Search}{Feb 21, 2025}

\begin{document}

\maketitle

\section{Intro}

\medskip The objective of line search: make the gradient algorithms work when you start far away from optimum.
These algorithms are sometimes called globalization strategies. For line search we have two elements: (i) conditions on step-length, Wolfe conditions and (ii) Step-length computation.


\medskip Line search iterates: $x_{k+1} = x_k + \alpha_k p_k$. Gradient descent directions: $p_k = -B_k^{-1} \nabla f(x_k) = -B_k^{-1}\nabla f_k$.
\begin{itemize}
  \item $B_k = I$: Steepest descent
  \item $B_k = \nabla ^2 f_k$: Newton
  \item $B_k \approx \nabla ^2 f_k$: Quasi-Newton
\end{itemize}

Observation: $B_k > 0 \implies p_k = -B_k^{-1} \nabla f_k$ descent direction.

Proof: $p_k^{\top} \nabla f_k = - \nabla f_k^{\top} B_k^{-1} \nabla f_k < 0$.   ($B_k > 0 \implies B_k^{-1} > 0$).

Note: Hessian is not necessarily positive definite (far from solution).

\paragraph{Topics today}- How to choose $\alpha_k$ and how to ensure $B_k > 0$.


\section{Descent}

Define $\phi(\alpha) = f(x_k + \alpha p_k)$. Here the input $\alpha$ determines how far we should walk along the direction $p_k$.

One strategy is to use exact linesearch. $\alpha^* = arg \min_{\alpha}\phi (\alpha)$. Not done in practice because $\alpha^*$ is too expensive to calculate, and not even necessary.
Instead, do inexact linesearch. Find `cheap' $\alpha$ that fulfills (i) sufficient decrease (Armijo), and (ii) Desired slope (curvature).


\medskip Why sufficient decrease? To make sure you actually make progress. Otherwise it is possible to keep going over the goal. This is the 1st Wolfe condition, or the Armijo condition. It says choose $\alpha$ that fulfills $\phi(\alpha) \leq \phi(0) + c_1 \alpha \phi(0) := l(\alpha)$
This condition allows for very small steps.

\medskip 2nd Wolfe condition: Desired slope. $\phi'(\alpha) > c_2 \phi'(0),\  c_2 \in(c_1, 1)$
The rationale is that we should not stop when $\phi(\alpha) << 0$. Typical value Newton/Q-N is $c_2 = 0.9$.

\paragraph{Summary}- Good step lengths should fulfill the Wolfe conditions:
\begin{itemize}
  \item $f(x_k + \alpha p_k) \leq f(x_k) + c_1\alpha_k \nabla f_k^{\top} p_k$
  \item $\nabla f (x_k + \alpha p_k)^{\top} p_k \geq c_2 \nabla f_k^{\top} p_k$
\end{itemize}

How do we compute the step length then?

\paragraph{Backtracking Line Search}- Algorithm 3.1

\lstset{basicstyle=\ttfamily}
\begin{lstlisting}
1. Choose @$\overline{\alpha} > 0,\ p \in (0, 1),\  c \in (0, 1)$@; Set @$\alpha \leftarrow \overline{\alpha}$@
2. repeat until @$f(x_k + \alpha p_k) \leq f(x_k) + c \alpha \nabla f_k^{\top}p_k$@
3.  @$\alpha \leftarrow p \alpha$@
4. end (repeat)
5. Terminate with @$\alpha_k = \alpha$@ 
\end{lstlisting}

Very easy to implement. A problem is how to choose $\overline{\alpha}$? In Newton/Q-N it is easy; start with $\overline{\alpha} = 1$ and reduce. But in steepest descent
there is no general way to pick a good value.

\paragraph{Interpolation}-

\medskip
\lstset{basicstyle=\ttfamily}
\begin{lstlisting}
1.  (1) Initialize with @$\alpha_0$@ (initial guess)
2.    If @$\alpha_0$@ fulfuills Wolfe -> exit
3.  (2) Quadratic Interpolation
4.     @$Q_q(\alpha) = a\alpha^2 + \phi'(0)\alpha_\tau \phi(0)$@
5.    @$a = (\phi(\alpha_0) - \phi(0) - \alpha_0 \phi'(0))/(\alpha_0^2)$@
6.    @$\alpha_1 = arg \min_{\alpha} \phi_q(\alpha)$@
7.    If @$\alpha_1$@ fulfills Wolfe -> exit
8.  (3) Do cubic Interpolation
9.    @$\phi_c(\alpha) = a\alpha^3 + b\alpha^2 + \phi'(0) \alpha + \phi(0)$@
10.   a, b can be found with expression from p. 58
11.   @$\alpha_2 = arg \min_\alpha \phi_c(\alpha)$@
12.   If @$\alpha_2$@ fulfills Wolfe -> exit
13. (4) Start over with @$\alpha_0 = \alpha_2$@
\end{lstlisting}

\paragraph{Example}- Line search for convex quadratic objective function.

\medskip Note: it does not really make sense to use line search for this problem, because it can be solved directly. This is
for demonstration purposes only. The problem is

\begin{align*}
  f(x) & = \frac{1}{2}x^{\top}Gx + c^{\top}x,\  G > 0 \\
       & x_k,\  p_k \text{ given}
\end{align*}

We get
\begin{align*}
  \phi(\alpha)  & = \frac{1}{2}(x_k + \alpha p_k)^{\top}G(x_k+\alpha p_k) + c^{\top}(x_k + \alpha p_k)          \\
                & = \frac{1}{2} p_k^{\top}G p_k \alpha^2 + x_k^{\top}G p_k \alpha + c^{\top} p_k \alpha + const \\
  \phi'(\alpha) & = p_k^{\top}Gp_k \alpha + (x_k^{\top}G+c^{\top})p_k = 0 \\ 
  \implies& \alpha^* = - \frac{(Gx_n+c)^{\top}p_k}{p_k^{\top}Gp_k} \\ 
  \text{If Newton: } & p_k = - [\nabla ^2 f_k]^{-1} \nabla f_k = -G^{-1}(Gx+c) \\ 
  \implies & \alpha^* = \frac{(Gx_k+ c)^{\top}G^{-1}(Gx_k +c)}{(Gx_k+c)^{\top}G^{-1}G G^{-1}(Gx_k+c)} = 1
\end{align*}

\section{Newton: Hessian modification}

\begin{align*}
  x_{k+1} = x_k + \alpha_k p_k,\  p_k = -[\nabla ^2 f(x_k)]^{-1} \nabla f(x_k)
\end{align*}

In practice we don't compute the inverse. Instead, just solve $\nabla ^2 f_k p_k = -\nabla f_k$. To solve such
a linear system, it is best to use Cholesky factorization. However, far from the solution, $\nabla ^2 f_k$ is typically not positive definite. Then we can 
also not use Cholesky (can use LDL). 

\medskip To fix these issues, modify the Hessian and replace $\nabla ^2 f_k$ with $\nabla ^2 f_k + E_k > 0 $. $E_k$ can be constructed in several ways. 
One example is $E_k = \tau_k I,\  \tau_k \left\{
  \begin{aligned}
    &0,\  \nabla ^2 f_k > 0 \\ 
    &-\lambda_{\min} (\nabla ^2 f_k) + \Delta,\  \text{otherwise}
  \end{aligned}
\right.$


\end{document}
