\documentclass{article}

\usepackage{graphicx}
\usepackage{amsmath}
\usepackage{amssymb}
\usepackage{amsthm}
\usepackage{listings}
\usepackage{xcolor}
\usepackage{tcolorbox}
\usepackage{caption}
\usepackage[hidelinks]{hyperref}
\setlength{\parindent}{0pt}
\usepackage[margin=0.5in]{geometry}
\usepackage{tikz}
\usepackage{pgfplots}
\pgfplotsset{compat=1.15}
\usepackage{mathrsfs}
\usetikzlibrary{arrows}
\usepackage{import}
\usepackage{float}
\usepackage{transparent}




% \usepackage{pstricks}      % does not work :(
% \usepackage{auto-pst-pdf}

\newcommand{\incfig}[2]{%
  \def\svgwidth{#2\columnwidth}%
  \raisebox{-\height}{\import{./figures/}{#1.pdf_tex}}%
}

\lstset{
  language=Matlab,
  basicstyle=\ttfamily\small,
  numbers=left,
  numberstyle=\tiny\color{gray},
  stepnumber=1,
  numbersep=5pt,
  backgroundcolor=\color[HTML]{e2e8f0},
  showspaces=false,
  showstringspaces=false,
  showtabs=false,
  frame=single,
  rulecolor=\color{black},
  captionpos=b,
  breaklines=true,
  breakatwhitespace=false,
  escapeinside={\%*}{*)},
  commentstyle=\color{green},
}


\newcommand{\Exercise}[1]{
  \begin{titlepage}
    \vbox{ }
    \vbox{ }
    \begin{center}
      \includegraphics[width=0.40\textwidth]{../img/NTNU_logo.png}\\[1cm]
      \textsc{\Large TTK4135 - Optimization and Control}\\[0.5cm]
      \vbox{ }
      { \huge \bfseries Exercise \##1}\\[0.4cm]

      \large
      \emph{Author:}\\
      Sondre Pedersen
      \vfill

      {\large\today}
    \end{center}
  \end{titlepage}
}

\newcommand{\Lecture}[2]{
  \title{\textbf{#1}}
  \date{#2}
}

\definecolor{framecolor}{HTML}{94a3b8}
\definecolor{bgcolor}{HTML}{cbd5e1}

\newtcolorbox{problem}[2][]{
  colback=bgcolor,
  colframe=framecolor,
  fonttitle=\bfseries\color{black},
  title=Problem #2,
  width=\textwidth,
  #1
}

% proof 
\definecolor{proofcolor}{HTML}{10b981}
\definecolor{proofbgcolor}{HTML}{d1fae5}

\newtcolorbox{proofbox}[2][]{
  colback=proofbgcolor,
  colframe=proofcolor,
  fonttitle=\bfseries\color{black},
  title=#2,
  width=\textwidth,
  sharp corners,
  #1
}

\newtheorem{theorem}{Theorem}

\newcommand{\thrm}[1]{
  \begin{center}
    \begin{minipage}[t]{0.6\textwidth}
      \begin{theorem}
        #1
      \end{theorem}
    \end{minipage}
  \end{center}
}


% Subtask 
\newcommand{\SUBTASK}[2]{\medskip\medskip\textbf{\large (#1) #2}\medskip}

% Matrix shorthand
\newcommand{\M}[1]{\begin{bmatrix} #1 \end{bmatrix}}

% Subheader for smaller sections
\newcommand{\subheader}[1]{\vspace{1em}\noindent\underline{\textbf{\large #1}}\vspace{0.5em}}


\lstset{basicstyle=\ttfamily, escapeinside={(*@}{@*)}}

\Lecture{Lecture 11: Practical use of MPC: Output feedback, target calculation and offset-free control}{Feb 13, 2025}

\begin{document}

\maketitle

\section{MPC controller - state feedback}

MPC controller algorithm:

\lstset{basicstyle=\ttfamily}
\begin{lstlisting}
1. For t = 0, 1, 2, ..., (*@$\infty$@*) do
2.  Determine (*@$x_t$@*)
3.  Solve open-loop opt. prob. with initial state (*@$x_0 = x_t$@*)
4.    (*@$\rightarrow u_0, u_1, u_2,\dots,u_{N-1}$@*)
5.  Set (*@$u_t = u_0$@*)
6. end
\end{lstlisting}

\begin{center}
  \incfig{block}{0.4}
\end{center}

Observation: MPC controller is non-linear state feedback. We determine $x_t$ with measurements, or state estimate.

\paragraph{Reference tracking (regulation)}

Say that we have a plant that we want to design an MPC for. A typical control objective is that we want controlled variables $\gamma_t \rightarrow \gamma_{ref}$ where
$\gamma_t = H x_t$. Note: Controlled variables $\gamma_t = H x_t$ may be different from measured variables $y_t = Cx_t$.

\medskip Steady state:
\begin{align*}
  x_s      & = Ax_s + Bu_s \rightarrow x_s = (I - A)^{-1}Bu_s \\
  \gamma_s & = Hx_s = H(I - A)^{-1}Bu_s                       \\
\end{align*}

Example:
\begin{align*}
  A        & = \M{0.8  & 0.4      \\ -0.1 & 0.9} ,\ B = \M{1 & 0.5 \\ 0 & 2},\  H = \M{1 & -1}  \\
  \gamma_s & = \M{3.33 & 8.33}u_s
\end{align*}

\medskip Observe: (i) Input constraints limit possible $\gamma_s$. $\M{0 \\ 0} \leq u \leq \M{1 \\ 1} \rightarrow 0 \leq \gamma_s  \leq 11.66$. With input constraints,
we cannot control our system to wherever we want. (ii) Several $u_s$ give same $\gamma_s$ (in this case). If $u_s = \M{0 \\ 0.24} \rightarrow \gamma_s = 2.0$. Also, $u_s = \M{0.17 \\ 0.17}  \rightarrow \gamma_s = 2.0$

\medskip The solution to this problem is target calculation. It can be formulated as a QP.
\begin{align*}
  \min\qquad        & \frac{1}{2}u_s^{\top}R_s u_s    \\
  \text{s.t.}\qquad & x_s = Axs + Bu_s                \\
                    & Hx_s = \gamma_{ref}             \\
                    & x^{low}  \leq x_s \leq x^{high} \\
                    & u^{low} \leq u_s  \leq u^{high}
\end{align*}

\section{Offset-free control}

Also known as integral control. This is a sort of robust control where you remove the effect of unknown 
disturbances. This i sthe rolw of `I' in PID control, which we don't have in MPC so far.
An unmodelled disturbance $d_t$ will give an offset in the controlled variable $\gamma_t$. Model with disturbance
\begin{align*}
  x_{t+1} &= Ax_t + Bu_t + A_d d_t \\ 
  y_t &= Cx_t + C_d d_t
\end{align*}

The disturbance model:
\begin{align*}
  d_{t+1} & = d_t 
\end{align*}

Idea: Augment the model in MPC to get
\begin{align*}
  \M{x_{t+1} \\ d_{t+1}} &= \M{A & A_d  \\ 0 & I}\M{x_t \\ d_t} + \M{B \\ 0}u_t \\ 
  y_t &= \M{c & c_d}\M{x_t \\ d_t} 
\end{align*}

Offset-free MPC: 
\begin{itemize}
  \item Use state estimator with the above model to estimate $\hat{x_t} and \hat{d_t}$
  \item Use the model above as model in MPC. 
\end{itemize}
Note:
\begin{itemize}
  \item Requires $(A_{a_1}, C_a)$ to be observable. A practical requirement to solve the problem of having
  to measure too many disturbances: $\dim( \hat{d_t}) \leq \dim(y_t)$
  \item Typical industrial practice: $A_d = 0,\  C_d = I$ (`bias update'). OFten works well, especially when 
  the process is slow. But it does not always work. 
  The advantage of this method is that you do not need a state estimator. 
  \item Target calculation must be modified to depend on $\hat{d}$. 
\end{itemize}

\end{document}
