\documentclass{article}

\usepackage{graphicx}
\usepackage{amsmath}
\usepackage{amssymb}
\usepackage{amsthm}
\usepackage{listings}
\usepackage{xcolor}
\usepackage{tcolorbox}
\usepackage{caption}
\usepackage[hidelinks]{hyperref}
\setlength{\parindent}{0pt}
\usepackage[margin=0.5in]{geometry}
\usepackage{tikz}
\usepackage{pgfplots}
\pgfplotsset{compat=1.15}
\usepackage{mathrsfs}
\usetikzlibrary{arrows}
\usepackage{import}
\usepackage{float}
\usepackage{transparent}




% \usepackage{pstricks}      % does not work :(
% \usepackage{auto-pst-pdf}

\newcommand{\incfig}[2]{%
  \def\svgwidth{#2\columnwidth}%
  \raisebox{-\height}{\import{./figures/}{#1.pdf_tex}}%
}

\lstset{
  language=Matlab,
  basicstyle=\ttfamily\small,
  numbers=left,
  numberstyle=\tiny\color{gray},
  stepnumber=1,
  numbersep=5pt,
  backgroundcolor=\color[HTML]{e2e8f0},
  showspaces=false,
  showstringspaces=false,
  showtabs=false,
  frame=single,
  rulecolor=\color{black},
  captionpos=b,
  breaklines=true,
  breakatwhitespace=false,
  escapeinside={\%*}{*)},
  commentstyle=\color{green},
}


\newcommand{\Exercise}[1]{
  \begin{titlepage}
    \vbox{ }
    \vbox{ }
    \begin{center}
      \includegraphics[width=0.40\textwidth]{../img/NTNU_logo.png}\\[1cm]
      \textsc{\Large TTK4135 - Optimization and Control}\\[0.5cm]
      \vbox{ }
      { \huge \bfseries Exercise \##1}\\[0.4cm]

      \large
      \emph{Author:}\\
      Sondre Pedersen
      \vfill

      {\large\today}
    \end{center}
  \end{titlepage}
}

\newcommand{\Lecture}[2]{
  \title{\textbf{#1}}
  \date{#2}
}

\definecolor{framecolor}{HTML}{94a3b8}
\definecolor{bgcolor}{HTML}{cbd5e1}

\newtcolorbox{problem}[2][]{
  colback=bgcolor,
  colframe=framecolor,
  fonttitle=\bfseries\color{black},
  title=Problem #2,
  width=\textwidth,
  #1
}

% proof 
\definecolor{proofcolor}{HTML}{10b981}
\definecolor{proofbgcolor}{HTML}{d1fae5}

\newtcolorbox{proofbox}[2][]{
  colback=proofbgcolor,
  colframe=proofcolor,
  fonttitle=\bfseries\color{black},
  title=#2,
  width=\textwidth,
  sharp corners,
  #1
}

\newtheorem{theorem}{Theorem}

\newcommand{\thrm}[1]{
  \begin{center}
    \begin{minipage}[t]{0.6\textwidth}
      \begin{theorem}
        #1
      \end{theorem}
    \end{minipage}
  \end{center}
}


% Subtask 
\newcommand{\SUBTASK}[2]{\medskip\medskip\textbf{\large (#1) #2}\medskip}

% Matrix shorthand
\newcommand{\M}[1]{\begin{bmatrix} #1 \end{bmatrix}}

% Subheader for smaller sections
\newcommand{\subheader}[1]{\vspace{1em}\noindent\underline{\textbf{\large #1}}\vspace{0.5em}}


\Lecture{Lecture 15: Quasi-Newton}{March 7, 2025}

\begin{document}

\maketitle

\section{Intro}

Quasi Newton efficiently produce good search directions. It needs fewer iterations than Steepest descent, and is cheaper than Newton.
This is the case because we use some cool techniques to approximate the Hessian.

\medskip
Conditions for a good step length: Wolfe Conditions
\begin{itemize}
  \item $f(x_k + \alpha p_k) \leq f(x_k) + c_1 + \alpha_k \nabla f_k^{\top}p_k\qquad$ Sufficient decrease (Armijo condition)
  \item $\nabla f(x_k + \alpha p_k)^{\top}p_k \geq c_2 \nabla f_k^{\top}p_k\qquad\qquad\qquad$ Desired slope (Curvature condition)
\end{itemize}

\paragraph{Newton's method}- recap

\medskip Approximate (`model') $f(x)$ at $x_k$.
\begin{align*}
  f(x_k + p) & \approx m_k(p) \\
             & = f_k + \nabla f_k^{\top} p + \frac{1}{2}p^{\top} \nabla ^2 f_k p
\end{align*}

The Newton direction: $p = arg \min_p m_k(p)$. If we assume that $\nabla ^2 f_k > 0 \implies m_k(p)$ convex. This implies $\nabla m_k(p) = 0$ necessary and sufficient condition for min. 
$\nabla m_k(p) = \nabla f_k + \nabla ^2 f_k p = 0 \implies p = -[\nabla ^2 f_k]^{-1}\nabla f_k$. The matrix is invertible because we assumed $\nabla ^2 f_k > 0$. 

\paragraph{Hessian modification}-

For $p_k = -B_k^{-1}\nabla f(x_k)$ to be a descent direction, we need $B_k > 0$. In general, this does not hold true for Newton, $B_k = \nabla ^2 f(x_k)$. We therefore modify the Hession when
it is not positive definite. This is done with algorithm 3.2. The problem with this method is that it uses expensive calculations. 

\section{Newton vs. Quasi-Newton}

\paragraph{Newton's method}- 

\medskip $x_{k+1} = x_k + \alpha_k p_k \qquad p_k = -[\nabla ^2 f_k]^{-1} \nabla f_k$.
\begin{itemize}
  \item Advantage: Fast convergence (few iterations). 
  \item Drawback: Expensive (at least for problems with many variables). Why is it expensive?  
  \begin{itemize}
    \item Calculating (and storing) the Hessian ($\nabla ^2 f_k$).
    \item Solve $\nabla ^2 f_k p_k = - \nabla f_k$. 
  \end{itemize}
\end{itemize}

\paragraph{Quasi-Newton}- 

\medskip Makes an approximation of $f(x)$ at $x_k$ with
\[
  f(x_k + p) \approx f_k + \nabla f_k^{\top}p + \frac{1}{2}p^{\top}B_k p 
.\] 

The difference between this and Newton is the therm $B_k$ instead of the Hessian. So the name of the game is to pick a 
good $B_k$. 


\medskip We want 
\begin{itemize}
  \item $B_k > 0 \qquad$ to ensure descent direction
  \item $B_k \approx \nabla ^2 f_k \qquad$ to ensure fast convergence. 
  \item Cheap computation, only using the gradient. 
\end{itemize}

\section{Secant condition}


Quasi-Newton was invented by Bill Davidon around the mid 1950s. Came up with `The Davidon-Fletcher-Powell' (DFP) update formula.
The key in this update rule was the secant condition. 

\medskip Consider 
\begin{align*}
  m_{k+1}(p) &= f_{k+1} + \nabla f_{k+1}^{\top}p + \frac{1}{2} p^{\top} B_k p \\ 
  \nabla m_{k+1}(p) &= \nabla f_{k+1} + B_k p   
\end{align*}

Now using the path ($\alpha p_k$) between $x_k$ and $x_{k+1}$ to evaluate the gradient at $x_{k+1}$:
\begin{itemize}
  \item $\nabla m_{k+1}(0) = \nabla f_{k+1}$
  \item $\nabla m_{k+1}(-\alpha p_k) = \nabla f_{k+1} - \alpha B_{k+1} p_k$
\end{itemize}

We want the second equation to equal $\nabla f_k$. To achieve this, $B_{k+1} \alpha_k p_k = \nabla f_{k+1} - \nabla f_k$. Defining some variables $s_k = x_{k+1} - x_k$ and $y_k = \nabla f_{k+1} - \nabla f_k$, 
the secant condition can be written as $B_{k+1} s_k = y_k$. 

\medskip Same comes from Taylor expansion of $\nabla f(x_k)$. 
\begin{align*}
  \nabla f_{k+1} &= \nabla f_k + \nabla ^2 f_k (x_{k+1}- x_k) + \dots
\end{align*}

That is, the secant condition implies $B_{k+1} \approx \nabla ^2 f(x_k)$. But it does not tell us how to compute $B_{k+1}$. 
We can use some of the requirements from earlier to do this.

\paragraph{Positive definite requirement}- 

\medskip We want $B_{k+1} > 0$. Note that $s_k^{\top}y_k = s_k^{\top}B_{k+1}s_k$. So we must require that 
\[
  s_k^{\top}y_k = (x_{k+1} - x_k)^{\top}(\nabla f_{k+1} - \nabla f_k) > 0
.\] 

This holds if the step length that we choose fulfills Wolfe condition. It also holds for any $\alpha$ if $f(x) $ is convex. 

\section{DFP update formula}

Observation: There are infinitely many $B_{k+1} > 0$ that fulfills $B_{k+1}s_k = y_k$. So we choose $B_{k+1}$ closest to $B_k$. 
\begin{align*}
  B_{k+1} &= \arg \min_B ||B-B_k|| \qquad\text{s.t.}\qquad B = B^{\top} ,\ Bs_k = y_k
\end{align*}

The norm you choose actually determines which Quasi-Newton method you get. The norm used in DFP is `weighted Frobenius norm':
\begin{align*}
  B_{k+1} =& (I - \rho_k y_k s_k^{\top})B_k (I - \rho_k s_k y_k^{\top}) + \rho_k y_k y_k^{\top} \\ 
  &\text{where } \\ 
   &\rho_k \frac{1}{y_k ^{\top}s_k} \\ 
   &y_k = \nabla f_{k+1} - \nabla f_k \\ 
   &s_k = x_{k+1} - x_k
\end{align*}

\textbf{However}, we need $B_k^{-1}$ in $p_k = -B_k^{-1} \nabla f_k$, can we update $H_k = B_k^{-1}$ instead?
Yes, just rewrite formula as 
\[
  B_{k+1} = B_k + \M{B_k s_k  & y_k}\M{0 & -\rho_k  \\ -\rho_k  & \rho_k + s_k^{\top}B_k s_k \rho_k^2}\M{s_k^{\top} B_k \\ y_k^{\top}}
.\] 

Use the matrix inversion lemma (Sherman-Morrison-Woodbury formula)
\[
  (A+UCV)^{-1} = A^{-1}- U(C^{-1}+ VA^{-1}U)^{-1}VA^{-1}
\] 

to obtain the inverse DFP formula:
\[
  H_{k+1} = H_k - \frac{H_k y_k y_k^{\top}H_k}{y_k^{\top}H_k y_k} + \frac{s_k s_k^{\top}}{y_k^{\top}s_k}
.\] 

\section{BFGS update formula}

As time went by, people agreed that BFGS update formula is better than DFP.
Alternatively: Choose $H_{k+1}$ closest to $H_k$: 
\[
  H_{k+1} = \arg \min_H ||H - H_k|| \qquad\text{s.t.}\qquad H = H^{\top} ,\ Hy_k = s_k
.\] 

Again we use the weighted Frobenius norm. Now the solution is the BFGS formula:
\[
  H_{k+1} = (I- \rho_k s_k y_k)^{\top}H_k (I-\rho_k y_k s_k^{\top}) + \rho_k s_k s_k^{\top}
.\] 

Very similar to the DFP, but considered the most effective Quasi-Newton formula. 
Note: $H_{k+1}$ positive definite if $y_k^{\top} s_k > 0$.

\medskip BFGS algorithm: 
\lstset{basicstyle=\ttfamily}
\begin{lstlisting}
Given starting point @$x_0$@, convergence tolerance @$\epsilon > 0$@,
  inverse Hessian approximation @$H_0$@;
k = 0;
while @$||\nabla f_k || > \epsilon$@; 
  Compute search direction 
  
    @$p_k = -H_k \nabla f_k$@; 
  Set @$x_{k+1} = x_k + \alpha_k p_k$@ where @$\alpha_k$@ is computed from a line search 
    procedure to satisfy the Wolfe conditions 
  Define @$s_k = x_{k+01} - x_k$@ and @$y_k = \nabla f_{k+1} - \nabla f_k$@; 
  Compue @$H_{k+1}$@ by means of (that long formula)
  k++
end
\end{lstlisting}






\end{document}
