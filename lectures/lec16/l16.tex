\documentclass{article}

\usepackage{graphicx}
\usepackage{amsmath}
\usepackage{amssymb}
\usepackage{amsthm}
\usepackage{listings}
\usepackage{xcolor}
\usepackage{tcolorbox}
\usepackage{caption}
\usepackage[hidelinks]{hyperref}
\setlength{\parindent}{0pt}
\usepackage[margin=0.5in]{geometry}
\usepackage{tikz}
\usepackage{pgfplots}
\pgfplotsset{compat=1.15}
\usepackage{mathrsfs}
\usetikzlibrary{arrows}
\usepackage{import}
\usepackage{float}
\usepackage{transparent}




% \usepackage{pstricks}      % does not work :(
% \usepackage{auto-pst-pdf}

\newcommand{\incfig}[2]{%
  \def\svgwidth{#2\columnwidth}%
  \raisebox{-\height}{\import{./figures/}{#1.pdf_tex}}%
}

\lstset{
  language=Matlab,
  basicstyle=\ttfamily\small,
  numbers=left,
  numberstyle=\tiny\color{gray},
  stepnumber=1,
  numbersep=5pt,
  backgroundcolor=\color[HTML]{e2e8f0},
  showspaces=false,
  showstringspaces=false,
  showtabs=false,
  frame=single,
  rulecolor=\color{black},
  captionpos=b,
  breaklines=true,
  breakatwhitespace=false,
  escapeinside={\%*}{*)},
  commentstyle=\color{green},
}


\newcommand{\Exercise}[1]{
  \begin{titlepage}
    \vbox{ }
    \vbox{ }
    \begin{center}
      \includegraphics[width=0.40\textwidth]{../img/NTNU_logo.png}\\[1cm]
      \textsc{\Large TTK4135 - Optimization and Control}\\[0.5cm]
      \vbox{ }
      { \huge \bfseries Exercise \##1}\\[0.4cm]

      \large
      \emph{Author:}\\
      Sondre Pedersen
      \vfill

      {\large\today}
    \end{center}
  \end{titlepage}
}

\newcommand{\Lecture}[2]{
  \title{\textbf{#1}}
  \date{#2}
}

\definecolor{framecolor}{HTML}{94a3b8}
\definecolor{bgcolor}{HTML}{cbd5e1}

\newtcolorbox{problem}[2][]{
  colback=bgcolor,
  colframe=framecolor,
  fonttitle=\bfseries\color{black},
  title=Problem #2,
  width=\textwidth,
  #1
}

% proof 
\definecolor{proofcolor}{HTML}{10b981}
\definecolor{proofbgcolor}{HTML}{d1fae5}

\newtcolorbox{proofbox}[2][]{
  colback=proofbgcolor,
  colframe=proofcolor,
  fonttitle=\bfseries\color{black},
  title=#2,
  width=\textwidth,
  sharp corners,
  #1
}

\newtheorem{theorem}{Theorem}

\newcommand{\thrm}[1]{
  \begin{center}
    \begin{minipage}[t]{0.6\textwidth}
      \begin{theorem}
        #1
      \end{theorem}
    \end{minipage}
  \end{center}
}


% Subtask 
\newcommand{\SUBTASK}[2]{\medskip\medskip\textbf{\large (#1) #2}\medskip}

% Matrix shorthand
\newcommand{\M}[1]{\begin{bmatrix} #1 \end{bmatrix}}

% Subheader for smaller sections
\newcommand{\subheader}[1]{\vspace{1em}\noindent\underline{\textbf{\large #1}}\vspace{0.5em}}


\Lecture{Lecture 16: Calculating derivatives and Derivative-free optimization}{March 13, 2025}

\begin{document}

\maketitle

\section{Intro}

The methods we have seen so far all need gradients. It is therefore worth it to see how to find these
derivatives as efficiently as possible. For small problems with 5 variables or 10 to 100 variables it is quite easy. But 
when you have a million variables, it becomes harder. Now we have to put some thought into it. 

\paragraph{Four ways of finding derivatives}- 
\begin{itemize}
  \item By hand 
  \begin{itemize}
    \item Time consuming 
    \item Error prone for large problems
  \end{itemize}
  \item Symbolic differentiation (CAS - computer algebra systems)
  \begin{itemize}
    \item  GeoGebra, Maple, \dots
    \item Expensive to evaluate, cumbersome to maintain.
  \end{itemize}
  \item Numerical differentiation
  \begin{itemize}
    \item Easy to implement, but may have low accuracy
  \end{itemize}
  \item Automatic (Algorithmic) Differentiation (AD)
  \begin{itemize}
    \item Best option when it can be done
    \item Easy to implement using the right software
    \item Exact up to machine precision
  \end{itemize}
\end{itemize}

\paragraph{Numerical differentiation}-

\medskip Single variable:
\[
  f: \mathbb{R} \rightarrow \mathbb{R} \qquad \frac{\partial f}{\partial x} \approx \frac{f(x + \epsilon) - f(x)}{\epsilon}    
.\]

Multi-variable (like objective functions) 
\[
  f: \mathbb{R}^{n} \rightarrow \mathbb{R} \qquad \text{directional derivative}
\] 
\[
  \nabla f(x)^{\top}p = \frac{\partial}{\partial \alpha} f(x + \alpha p) \big|_{\alpha = 0} \approx \frac{f(x+ \epsilon p) - f(x)}{\epsilon}
.\] 

If we want to find the partial derivative of f,

\[
  \frac{\partial f}{\partial x_i} \approx \frac{f(x + \epsilon e_i) - f(x)}{\epsilon}
.\] 

This is the special case of finding direcitonal derivative along one of the coordinates. 


To get the full gradient $\nabla f(x)$, iterate over all the coordinates. 

\section{Theoretical accuracy of numerical differentiation}

$f: \mathbb{R}^{n} \rightarrow \mathbb{R}$, assume $||\nabla ^2 f(x)|| \leq L$

\medskip Taylor: 
\begin{align*}
  &|f(x+p) - f(x) - \nabla f(x)^{\top}p| = | \frac{1}{2}p^{\top} \nabla ^2 f(x+tp) p| \\ 
  & \leq \frac{1}{2}||\nabla ^2 f(x+tp)|| \times ||p||^2 \leq \frac{L}{2}||p||^2   
\end{align*}

Let $p = \epsilon e_i$:
\begin{align*}
  &f(x+\epsilon e_i) - f(x) - \epsilon \nabla f(x)^{\top}e_i|  \leq \frac{L}{2}\epsilon ^2 \\ 
  &| \frac{f(x+\epsilon e_i) - f(x)}{\epsilon} - \frac{\nabla f}{\nabla x_i}| \leq \frac{L}{2}\epsilon  \\ 
\end{align*}

In theory, we can just pick an $\epsilon$ small enough to get any accuracy we want (not in practice!). 
These difference schemes can be cooked up to give better accuracy for the same $\epsilon$. For example, central differences
is much better than just one-sided difference. 

\medskip The limit on how small you can make $\epsilon$ and still improve accuracy is determined by how small the numbers the computer can 
represent. A normal limit would be around $10^{-8}$ before you would run into problems. 


\paragraph{Approximating the Hessian}- 

\medskip In many cases, the gradient is available, but not the Hessian. We can use finite differences on the gradient:
\[
  \nabla ^2 f(x) p \approx \frac{\nabla f(x + \epsilon p) - \nabla f(x)}{\epsilon}
.\]   

If the gradient is not available, use finite differences `twice' for Hessian. But this is terrible, so use Quasi-Newton instead. 

\section{Automatic (algorithmic) differentiation (AD)}

These are software tools that automatically computes derivatives of your code. The principle being used is automated chain rule. 
It is possible to calculate derivatives forward and in reverse. The reverse mode sounds strange, but turns out to be efficient in some
important applications. 

\medskip Today AD is implemented using operator overloading. It requires that our implementation is differentiable. These methods are 
used in all machine learning tools. 

\paragraph{AD}- forward and reverse

\medskip In forward mode, both $x_i$ and $\nabla x_i$ are calculated by forward traversing computation graph. 
\[
  f(x_1, x_2, x_3) = x_1x_2+x_3 \cos{(x_1)} + x_3
.\] 


\end{document}
