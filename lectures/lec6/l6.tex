\documentclass{article}

\usepackage{graphicx}
\usepackage{amsmath}
\usepackage{amssymb}
\usepackage{amsthm}
\usepackage{listings}
\usepackage{xcolor}
\usepackage{tcolorbox}
\usepackage{caption}
\usepackage[hidelinks]{hyperref}
\setlength{\parindent}{0pt}
\usepackage[margin=0.5in]{geometry}
\usepackage{tikz}
\usepackage{pgfplots}
\pgfplotsset{compat=1.15}
\usepackage{mathrsfs}
\usetikzlibrary{arrows}
\usepackage{import}
\usepackage{float}
\usepackage{transparent}




% \usepackage{pstricks}      % does not work :(
% \usepackage{auto-pst-pdf}

\newcommand{\incfig}[2]{%
  \def\svgwidth{#2\columnwidth}%
  \raisebox{-\height}{\import{./figures/}{#1.pdf_tex}}%
}

\lstset{
  language=Matlab,
  basicstyle=\ttfamily\small,
  numbers=left,
  numberstyle=\tiny\color{gray},
  stepnumber=1,
  numbersep=5pt,
  backgroundcolor=\color[HTML]{e2e8f0},
  showspaces=false,
  showstringspaces=false,
  showtabs=false,
  frame=single,
  rulecolor=\color{black},
  captionpos=b,
  breaklines=true,
  breakatwhitespace=false,
  escapeinside={\%*}{*)},
  commentstyle=\color{green},
}


\newcommand{\Exercise}[1]{
  \begin{titlepage}
    \vbox{ }
    \vbox{ }
    \begin{center}
      \includegraphics[width=0.40\textwidth]{../img/NTNU_logo.png}\\[1cm]
      \textsc{\Large TTK4135 - Optimization and Control}\\[0.5cm]
      \vbox{ }
      { \huge \bfseries Exercise \##1}\\[0.4cm]

      \large
      \emph{Author:}\\
      Sondre Pedersen
      \vfill

      {\large\today}
    \end{center}
  \end{titlepage}
}

\newcommand{\Lecture}[2]{
  \title{\textbf{#1}}
  \date{#2}
}

\definecolor{framecolor}{HTML}{94a3b8}
\definecolor{bgcolor}{HTML}{cbd5e1}

\newtcolorbox{problem}[2][]{
  colback=bgcolor,
  colframe=framecolor,
  fonttitle=\bfseries\color{black},
  title=Problem #2,
  width=\textwidth,
  #1
}

% proof 
\definecolor{proofcolor}{HTML}{10b981}
\definecolor{proofbgcolor}{HTML}{d1fae5}

\newtcolorbox{proofbox}[2][]{
  colback=proofbgcolor,
  colframe=proofcolor,
  fonttitle=\bfseries\color{black},
  title=#2,
  width=\textwidth,
  sharp corners,
  #1
}

\newtheorem{theorem}{Theorem}

\newcommand{\thrm}[1]{
  \begin{center}
    \begin{minipage}[t]{0.6\textwidth}
      \begin{theorem}
        #1
      \end{theorem}
    \end{minipage}
  \end{center}
}


% Subtask 
\newcommand{\SUBTASK}[2]{\medskip\medskip\textbf{\large (#1) #2}\medskip}

% Matrix shorthand
\newcommand{\M}[1]{\begin{bmatrix} #1 \end{bmatrix}}

% Subheader for smaller sections
\newcommand{\subheader}[1]{\vspace{1em}\noindent\underline{\textbf{\large #1}}\vspace{0.5em}}


\Lecture{Lecture 6: Quadratic Programming \& Equality-constrained QPs}{January 24, 2025}

\begin{document}

\maketitle

\begin{minipage}[c]{0.3\textwidth}
  \raggedright
  \begin{itemize}
    \item Quadratic programming
          \begin{itemize}
            \item Convex problem  if $G \geq 0$
            \item Feasible set polyhedron
          \end{itemize}
  \end{itemize}
\end{minipage}
\begin{minipage}[c]{0.3\textwidth}
  \centering
  \begin{align*}
    \min_{x\in \mathbb{R}^{n}} \quad & \frac{1}{2}x^{\top}Gx + q^{\top}x                    \\
    \text{s.t.} \quad                & c_i(x) = a_i^{\top}x - b_i = 0,\ i\in \mathcal{E}    \\
                                     & c_i(x) = a_i^{\top}x -b_i \geq 0,\  i\in \mathcal{I}
  \end{align*}
\end{minipage}
\begin{minipage}[c]{0.3\textwidth}
  \raggedleft
  \incfig{quadratic}{0.8}
\end{minipage}

The easy case is when $G \geq 0$, because the QP is convex. When $G \ngeq 0$, then the problem is non-convex, and much harder.

\medskip Quadratic programming is common in control. Quadratic programming also shows up in production optimization when the price varies with ammount produced. This is a
realistic example. Here is an example comparing linear profit with profit that depends on how much you produce. :

\begin{minipage}[c]{0.5\textwidth}
  \begin{align*}
    \max_{x_1, x_2} \qquad 7000 x_1 + 6000x_2 &             \\
    \text{s.t.} \qquad 4000x_1 + 3000x_2      & \geq 100000 \\
    60x_1 + 80x_2                             & \leq 2000   \\
    x_1                                       & \geq 0      \\
    x_2                                       & \geq 0      \\
  \end{align*}
  \begin{center}
    \incfig{lp}{0.8}
  \end{center}
\end{minipage}
\begin{minipage}[c]{0.5\textwidth}
  \begin{align*}
    \max_{x_1, x_2} \qquad (7000-200x_1)x_1 + (6000 & - 140x_2)x_2 \\
    \text{s.t.} \qquad 4000x_1 + 3000x_2            & \leq 100000  \\
    60x_1 + 80x_2                                   & \leq 2000    \\
    x_1                                             & \geq 0       \\
    x_2                                             & \geq 0       \\
  \end{align*}
  \begin{center}
    \incfig{qp}{0.8}
  \end{center}
\end{minipage}

\medskip
The level curves are no longer linear, but real curves. Also, the optimal point may no longer lie in the corner, which is the property used in LP solvers. 

\medskip Rewriting the objective function for the quadratic version, we get
\begin{align*}
  &\min_{x_1, x_2} 200x_1^2 + 140x_2^2 - 7000x_1 - 6000x_2 \\
  &\min_{x\in \mathbb{R}^{2}} \frac{1}{2}x^{\top}\M{400 & 0 \\ 0 & 280}x + \M{-7000 & -6000}x
\end{align*}

The matrix here is G, and it is always symmetric. 

\section{Equality-constrained QPs}

In general
\begin{align*}
  \min_{x\in\mathbb{R}^{n}} \qquad& \frac{1}{2}x^{\top}Gx + c^{\top}x \\
  \text{s.t.}\qquad & a_i^{\top}x - b_i = 0,\  i\in \mathcal{E} = \{1,,\dots,m\},\ \mathcal{I} = Ø
\end{align*}

Define $A = \M{a_1^{\top}  \\ a_2^{\top}  \\  \vdots  \\ a_m^{\top}}\in \mathbb{R}^{m\times n},\ m < n$. $b = \M{b_1 \\ b_2 \\ \vdots \\ b_m}$ and we assume $rank(A) = m$ (full row rank).
Now we can write it (EQP) as
\begin{align*}
  \min_{x\in \mathbb{R}^{n}} \qquad& \frac{1}{2}x^{\top}Gx + c^{\top}x \\
  \text{s.t.} \qquad& Ax = b
\end{align*}  

\paragraph{KKT} for equality-constrained QP

Since we have equality constraints, only the two KKT conditions need to be considered. The lagrangian:
\[
  \mathcal{L}(x, \lambda) = \frac{1}{2}x^{\top}Gx + c^{\top}x - \lambda^{\top}(Ax-b) 
.\] 

KKT: 
\begin{align*}
  \nabla_{x}\mathcal{L}(x^*, \lambda^*) &= Gx^*+c - A^{\top}\lambda^* = 0 \\
  Ax^* &= b
\end{align*}

The KKT conditions are linear here, so we can write them in matrix form.
\[
  \M{G & -A^{\top} \\ A & 0}\M{x^* \\ \lambda^*} = \M{-c \\ b}
.\] 
Solving this system will solve the KKT condition. An alternative form that can be more practical ($x^* = x + p$):
\[
  \M{G & A^{\top} \\ A & 0}\M{-p \\ \lambda^*} = \M{c-Gx^* \\ Ax-b}
.\] 

These are called KKT-systems, and they show up a lot. 

\paragraph{Nullspace} is needed to know when solution to KKT is also the solution to EQP. 

\medskip Given feasible x (Ax = b). Rewrite the EQP as
\begin{align*}
  \min_p \qquad& \frac{1}{2}(x+p)^{\top}G(x+p) + x^{\top}(x+p)  \\ 
  \text{s.t.} \qquad& A(x+p) = b \qquad \text{(Ap = 0 since Ax = b)}
\end{align*}
We are searching for solutions p in $Null(A) = \{w | Aw = 0\}$. 

\medskip Let columns of $Z\in \mathbb{R}^{n\times (n-m)}$ span $Null(A)$. Use QR to find this. 

\medskip Example: $A_p = \M{1 & 0 & 0}\M{p_1 \\ p_2 \\ p_3}= 0 \qquad 
\left(
  \begin{aligned}
    &m = 1  \\ 
    &n = 3
  \end{aligned}
\right)$

$A\M{0 \\ 1 \\ 0} = 0,\ A\M{0 \\ 0 \\ 1} = 0 \implies Z = \M{0 & 0 \\ 1 & 0 \\ 0 & 1} \implies AZ = 0$

\paragraph{When can EQP be solved}- Assume A full row rank. Also, $Z^{\top}GZ > 0$, then $\M{G & A^{\top} \\ A & S}$ is non-singluar. This implies $\left\{
  \begin{aligned}
    x^*& = x+p \qquad \text{is unique solution}  \\ 
    \lambda^*& \qquad \text{to KKT system} 
  \end{aligned}
\right.$

\thrm{
  \[
    \left.
      \begin{aligned}
          &\text{A full row rank}  \\ 
          &Z^{\top}GZ > 0 
      \end{aligned}
    \right\} \implies x^* \text{ unique solution to EQP}
  .\] 
}

\begin{proofbox}{Proof Theorem 1 (16.2)}
A x is feasible, $x^*$ is a solution to KKT-system, $x+p = x^*$. 
\begin{align*}
  &Ap = A(x^*-x)  = Ax^*-Ax = b - b = 0 \implies p\in Null(A)
\end{align*}
We want to show $q(x) > q(x^*),\  x \neq x^*$. 
\begin{align*}
  q(x) &= \frac{1}{2}x^{\top}Gx + x^{\top}x = \frac{1}{2}(A^*- p) + c^{\top}(x^*-p)  \\ 
  &= \frac{1}{2}x^{*\top}GA^* + \frac{1}{2}p^{\top}Gp + c^{\top}x^* - c^{\top}p  \\ 
  &= q(x^*) - p^{\top}(GA^*+c) + \frac{1}{2}p^{\top}Gp  \\ 
  &= q(x^*) + \frac{1}{2}u^{\top}Z^{\top}GZu  \\ 
  & > q(x^*)
\end{align*}
\end{proofbox}

\paragraph{Nullspace method/Elimination of varialbes}- ?





\end{document}
