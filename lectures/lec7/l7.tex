\documentclass{article}

\usepackage{graphicx}
\usepackage{amsmath}
\usepackage{amssymb}
\usepackage{amsthm}
\usepackage{listings}
\usepackage{xcolor}
\usepackage{tcolorbox}
\usepackage{caption}
\usepackage[hidelinks]{hyperref}
\setlength{\parindent}{0pt}
\usepackage[margin=0.5in]{geometry}
\usepackage{tikz}
\usepackage{pgfplots}
\pgfplotsset{compat=1.15}
\usepackage{mathrsfs}
\usetikzlibrary{arrows}
\usepackage{import}
\usepackage{float}
\usepackage{transparent}




% \usepackage{pstricks}      % does not work :(
% \usepackage{auto-pst-pdf}

\newcommand{\incfig}[2]{%
  \def\svgwidth{#2\columnwidth}%
  \raisebox{-\height}{\import{./figures/}{#1.pdf_tex}}%
}

\lstset{
  language=Matlab,
  basicstyle=\ttfamily\small,
  numbers=left,
  numberstyle=\tiny\color{gray},
  stepnumber=1,
  numbersep=5pt,
  backgroundcolor=\color[HTML]{e2e8f0},
  showspaces=false,
  showstringspaces=false,
  showtabs=false,
  frame=single,
  rulecolor=\color{black},
  captionpos=b,
  breaklines=true,
  breakatwhitespace=false,
  escapeinside={\%*}{*)},
  commentstyle=\color{green},
}


\newcommand{\Exercise}[1]{
  \begin{titlepage}
    \vbox{ }
    \vbox{ }
    \begin{center}
      \includegraphics[width=0.40\textwidth]{../img/NTNU_logo.png}\\[1cm]
      \textsc{\Large TTK4135 - Optimization and Control}\\[0.5cm]
      \vbox{ }
      { \huge \bfseries Exercise \##1}\\[0.4cm]

      \large
      \emph{Author:}\\
      Sondre Pedersen
      \vfill

      {\large\today}
    \end{center}
  \end{titlepage}
}

\newcommand{\Lecture}[2]{
  \title{\textbf{#1}}
  \date{#2}
}

\definecolor{framecolor}{HTML}{94a3b8}
\definecolor{bgcolor}{HTML}{cbd5e1}

\newtcolorbox{problem}[2][]{
  colback=bgcolor,
  colframe=framecolor,
  fonttitle=\bfseries\color{black},
  title=Problem #2,
  width=\textwidth,
  #1
}

% proof 
\definecolor{proofcolor}{HTML}{10b981}
\definecolor{proofbgcolor}{HTML}{d1fae5}

\newtcolorbox{proofbox}[2][]{
  colback=proofbgcolor,
  colframe=proofcolor,
  fonttitle=\bfseries\color{black},
  title=#2,
  width=\textwidth,
  sharp corners,
  #1
}

\newtheorem{theorem}{Theorem}

\newcommand{\thrm}[1]{
  \begin{center}
    \begin{minipage}[t]{0.6\textwidth}
      \begin{theorem}
        #1
      \end{theorem}
    \end{minipage}
  \end{center}
}


% Subtask 
\newcommand{\SUBTASK}[2]{\medskip\medskip\textbf{\large (#1) #2}\medskip}

% Matrix shorthand
\newcommand{\M}[1]{\begin{bmatrix} #1 \end{bmatrix}}

% Subheader for smaller sections
\newcommand{\subheader}[1]{\vspace{1em}\noindent\underline{\textbf{\large #1}}\vspace{0.5em}}


\Lecture{Lecture 7: Active Set Method for Quadratic Programming}{January 30, 2025}

\begin{document}

\maketitle

\section{Background for Active Set Method}

Simplified preview:
\medskip \begin{itemize}
  \item Make a guess of which inequality constraints are active
  \item Solve corresponding EQP
  \item Check KKT-conditions
        \begin{itemize}
          \item If KKT OK: done
          \item else: update active contraint guess and go to 2.
        \end{itemize}
\end{itemize}

\paragraph{KKT for QP} $\min_{x\in\mathbb{R}^{n}} \frac{1}{2} x^{\top}Gx + c^{\top}x \qquad \text{s.t.}\qquad\left\{
  \begin{aligned}
    a_i^{\top}x & = b_i,\  i\in\mathcal{E}    \\
    a_i^{\top}x & \geq b_i ,\ i\in\mathcal{I}
  \end{aligned}
  \right.$

Lagrangian:
\[
  \mathcal{L}(x, \lambda) = \frac{1}{2}x^{\top}Gx + x^{\top}x - \sum_{i \in\mathcal{E}\cup\mathcal{I}} \lambda_i(a_i^{\top}x-b_i)
  .\]
KKT:
\begin{align*}
  Gx^*+ c - \sum_{i \in\mathcal{E}\cup\mathcal{I}}\lambda_i^*a_i & = 0                                  \\
  a_i^{\top}x^*                                                  & = b_i,\ i\in\mathcal{E}              \\
  a_i^{\top}x^*                                                  & \geq b_i,\  i\in \mathcal{I}         \\
  \lambda_i^*                                                    & \geq ,\ i\in \mathcal{I}             \\
  \lambda_i ^*(a_i^{\top}x-b_i)                                  & = 0, i\in \mathcal{I}\cup\mathcal{E}
\end{align*}

To make active set method, we reformulate KKT via the active set, which was defined as:
\begin{align*}
  \mathcal{A}(x^*) = \{i\in \mathcal{E}\cup\mathcal{I} | a_i^{\top}x^* = b_i\}
\end{align*}

Assume we know $\mathcal{A}(x^*)$, then KKT is
\begin{align*}
  Gx^* + x - \sum_{\mathcal{A}(x^*)}\lambda_i^*a_i & = 0                                                      \\
  a_i^{\top}x^*                                    & = b_i,\  i\in \mathcal{A}(x^*)                           \\
  a_i^{\top}x^*                                    & \geq b_i,\  i\in \mathcal{I} \backslash \mathcal{A}(x^*) \\
  \lambda_i^*                                      & \geq 0 ,\ i\in \mathcal{I}\cup\mathcal{A}(x^*)
\end{align*}

The lambdas that are not active equals 0.

\begin{proofbox}{Theorem 16.4: If $x^*$ satisifies KKT and $G \geq 0$ then $x^*$ is a global solution}
  Proof: Assume x is feasible, $x \neq x^*$. Note first: $(x-x^*)^{\top}(Gx^*+ c) = (x-x^*)^{\top}\sum_{i\in \mathcal{A}(x^*)}\lambda_i^*a_i = \sum_{i\in \mathcal{E}}\lambda_i^*a_i^{\top}(x-x^*) + \sum_{i\in \mathcal{A}(x^*) \cup\mathcal{I}}\lambda_i^*a_i^{\top}(x-x^*)$

  \medskip \begin{align*}
    q(x) & = \frac{1}{2}(x^*+ (x-x^*)^{\top})G(x^*+ (x-x^*))+c^{\top}(x^*+(x-x^*))                           \\
         & = \frac{1}{2}x^{*\top}Gx^*+c^{\top}x^*+ (x-x^*)^{\top}G(x-x^*) + (x-x^*)^{\top}Gx+c^{\top}(x-x^*) \\
         & = q(x^*) + \frac{1}{2}(x-x^*)^{\top}G(x-x^*)+(x-x^*)^{\top}(Gx^* + c)                             \\
         & \geq q(x^*)\qquad\text{proof complete}
  \end{align*}

  \medskip Note: $Z^{\top}GZ > 0 \implies q(x) > q(x^*)$, that is: $x^*$ is strict global solution.
\end{proofbox}

\medskip If we know $\mathcal{A}(x^*)$, QP can be forumalted as
\[
  \min_{xin\mathbb{R}^{n}} \frac{1}{2}x^{\top}Gx + c^{\top}x \qquad\text{s-t-}\qquad a_i^{\top}x = b_i ,\ i\in \mathcal{A}(x^{\top})
  .\]

This can then be solved by solving the KKT-system:
\[
  \M{G & -A^{\top} \\ A & 0}\M{x^* \\ \lambda^*} = \M{-c \\ b} \left\{
  \begin{aligned}
    A & = \M{\vdots \\ a_i^{\top} \\ \vdots} ,\ i\in \mathcal{A}(x^*)  \\
    b & = \M{\vdots \\ b_i \\ \vdots} ,\ i\in \mathcal{A}(x^*)
  \end{aligned}
  \right.
  .\]

\section{One step of active set method for QP}

Iteration k:
\begin{itemize}
  \item $\mathcal{W}_k$: current `guess' of $\mathcal{A}(x^*)$.
  \item $x_k$: current feasible estimate of $x^*$.
  \item $g_k$: $(Gx_k + c)^{\top}$
\end{itemize}

Define $A_x = \M{\vdots \\ Q_i^{\top} \\ \vdots}, i\in \mathcal{W}_k,\  b_k = \M{\vdots \\ b_i \\ \vdots}, i\in \mathcal{W}_k$

\medskip Consider EQP: $x = x_k + p$:
\begin{align*}
  \min_{p\in \mathbb{R}^{n}}\quad           & \frac{1}{2}(x_k+ p) ^{\top}G(x_k + p) + c^{\top}(x_k+ p)                  \\
  \text{s.t.}\quad                          & A_k(x_k + p) = b_k                                                        \\
  \implies \min_{p\in \mathbb{R}^{4}} \quad & \frac{1}{2}p^{\top}Gp + (Gx_k + c)^{\top}p \qquad\text{s.t.}\qquad Ap = 0
\end{align*}

\begin{itemize}
  \item If $p_k = 0$: solve $\sum_{i\in \mathcal{W}_k}a_i \hat{\lambda_i} = g_k$ for $\hat{\lambda}$ ($A_k \hat{\lambda} = g_k$)
        \begin{itemize}
          \item If $\hat{\lambda} \geq 0$:
                \begin{itemize}
                  \item All KKT-conditions fulfulled
                  \item $x_k$ is our solution
                \end{itemize}
          \item If $\hat{\lambda} \ngeq 0$:
                \begin{itemize}
                  \item Pick index of one negative $\hat{\lambda_i}$
                  \item Remove this index from $\mathcal{W}_k$
                  \item Start over.
                \end{itemize}
        \end{itemize}
  \item If $p_k \neq 0$: 
  \begin{itemize}
    \item If $x_{k+1} = x_k + p_k$ is feasible: Set $\mathcal{W}_{k+1} = \mathcal{W}_k$, start over
    \item If $x_{k+1} = x_k + p_k$ is not feasible: find blocking constraint 
    \begin{itemize}
      \item For $i\in \{i | a_i^{\top}p_k < 0\}$ 
      \item want $a_i^{\top}(x_k+\alpha_ip_k) \geq b_i$ 
      
      $\qquad \implies \lambda_i = \frac{b_i - a_i^{\top}x_k}{\alpha_i^{\top}p_k}$
      \item Set j = "i with smallest $\alpha_i"$
      \item $x_{k + 1} = x_n + \alpha_j p_k$, $\mathcal{W}_{k+1} = \mathcal{W}_k + \{j\}$
      \item start over\dots
    \end{itemize}
  \end{itemize}
\end{itemize}

This is one of the key algorithms in the course. 

\paragraph{Example 16.4}-

\medskip \begin{minipage}[c]{0.5\textwidth}
  \centering
  \incfig{ex16}{1}
\end{minipage}
\begin{minipage}[c]{0.5\textwidth}
  \begin{align*}
    \min_x q(x) = (x_1-1)^2 + (x_2-2.5)^2& \\ 
    \text{s.t.}\quad x_1 -2x_2+2 \geq 0 \qquad(1)& \\ 
    -x_1-2x_2+6 \geq 0 \qquad(2)& \\ 
    -x_1+2x_2+2  \geq 0 \qquad(3)& \\ 
    x_1 \geq 0 \qquad(4)&  \\ 
    x_2 \geq 0 \qquad(5) 
  \end{align*}
\end{minipage}

\begin{minipage}[t]{0.2\textwidth}
  \begin{align*}
    G &= \M{2 & 0 \\ 0 & 2}  \\ 
    a_1 &=\M{1 & -2}^{\top} \\ 
    a_2 &=\M{-1 & -2}^{\top} \\ 
    a_3 &=\M{-1 & 2}^{\top} \\ 
    a_4 &=\M{1 & 0}^{\top} \\ 
    a_5&=\M{0 & 1}^{\top} 
  \end{align*}
\end{minipage}
\begin{minipage}[t]{0.2\textwidth}
  \begin{align*}
    c &= \M{-2 \\ -5} \\ 
    b_1 &= -2 \\ 
    b_2 & =-6 \\ 
    b_3&= -2 \\ 
    b_4 &= 0 \\ 
    b_5 &= 0
  \end{align*}  
\end{minipage}

We have two different indices, one for iteration and one for variable. Denote iteration k like $x^k$.

\medskip $x^{0} = \M{2 \\ 0},\ \mathcal{W}= \{3, 5\} ,\ g_k = Gx^{0}+ c = \M{2 \\ -5}$

\medskip EQP:
\begin{align*}
  &\min_p \frac{1}{2}p^{\top}Gp + g_0^{\top}p  \\ 
  &\text{s.t. } A_0 p = 0 \\  \\ 
  &\min_p p_1^2+p_2^2+2p_1-5p_2 \\ 
  &\text{s.t. }-p_1+2p_2 = 0 \\ 
  &p_2 = 0  \\  \\ 
  & \sum_{i\in \mathcal{W}_0}a_i \hat{\lambda_i} = g_0 : \M{-1 \\ -2} \hat{\lambda_3} + \M{0 \\ 1} \hat{\lambda_5} = \M{2 \\ -5}
\end{align*}

Remove \{3\} from $\mathcal{W}_0$. Continue with next iteration.
We have $x^{1} = \M{2 \\ 0},\ \mathcal{W}_1 = \{5\},\ g_1 = \M{2 \\ -5}$

EQP gives $p^{1} = \M{-1 \\ 0}$. 

\medskip $x^2 = x^{1}+ p ^{1} = \M{1 \\ 0}$, which is feasible. $\mathcal{W}_2 = \mathcal{W}^{1}$

And some more stuff

\end{document}
