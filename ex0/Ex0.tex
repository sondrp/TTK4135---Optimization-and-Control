\documentclass{article}
\usepackage{graphicx}
\usepackage{amsmath}
\usepackage{amssymb}
\usepackage{listings}
\usepackage{xcolor}
\usepackage{tcolorbox}
\usepackage{caption}
\usepackage[hidelinks]{hyperref}
\setlength{\parindent}{0pt}
\usepackage[margin=0.5in]{geometry} % Set all margins to 1 inch



% Define custom colors
\definecolor{framecolor}{HTML}{94a3b8}
\definecolor{bgcolor}{HTML}{cbd5e1}

\newtcolorbox{problem}[2][]{
  colback=bgcolor,
  colframe=framecolor,
  fonttitle=\bfseries\color{black},
  title=Problem #2,
  width=\textwidth,
  #1
}

\begin{document}

\input{../template/framside.tex}
\setExerciseNumber{0}

\begin{problem}{1: Definitions}
(a) What is the definition of the gradient of a scalar function $\textit{f} : \mathbb{R}^{n} \to \mathbb{R}$? Is it a row or column vector?

\medskip

(b) What is the definition of the Jacobian of a function $\textit{f} : \mathbb{R}^{n} \to \mathbb{R}^{m}$? If m = 1, what is the difference between the gradient and the Jacobian?
\end{problem}

\begin{problem}{2: Linear functions}
Let $\mathbf{f(x)} = \mathbf{Ax}$ where
\[
  \mathbf{A} = 
  \begin{bmatrix}
    a_{11} & a_{12} \\
    a_{21} & a_{22}
  \end{bmatrix}
  ,\ 
  \mathbf{x} = 
  \begin{bmatrix}
    x_1 \\ x_2
  \end{bmatrix}
  .\]

(a) Use the definition and calculate $\frac{\partial\mathbf{f(x)}}{\partial\mathbf{x}}$. Is this the Jacobian or the gradient of $\mathbf{f(x)}$?

\medskip

(b) Can you, without doing any calculations, find $\frac{\partial\mathbf{Ax}}{\partial\mathbf{x}}$ when $\mathbf{x}$ is a 
column vector of length \textit{n}, and $\mathbf{A}$ is a matrix of dimension $m\times n$ (i.e., \textit{m} rows and \textit{n})?
\end{problem}

\begin{problem}{3: Nonlinear/quadratic functions}
Let $f(x,y)=\mathbf{x^{\top}Gy}$, where
\[
\mathbf{x} = \begin{bmatrix}
  x_1 \\ x_2
\end{bmatrix}
,\ 
\mathbf{G} = \begin{bmatrix}
  g_{11} & g_{12} & g_{13} \\
  g_{21} & g_{22} & g_{23} \\
\end{bmatrix}
,\ 
\mathbf{y} = \begin{bmatrix}
  y_1 \\ y_2 \\ y_3\\
\end{bmatrix}
.\] 

(a) What is the dimensions of $f(x,y)$? Is $\nabla f(x,y)$ equal to $\frac{\partial f(x,y)}{\partial x}$ (no calculations are needed)?

\medskip

(b) Use the definition and calcualte $\nabla_x f(x,y)$. Then write the answer in matrix form.

\medskip

(c) Use the definition and calcualge $\nabla_y f(x,y)$. Then write the answer in matrix form.

\medskip

(d) Let $f(x) = x^{\top}\mathbf{Hx}$, where $x \in \mathbb{R}^{n}$ and $\mathbf{H} \in \mathbb{R}^{n \times n}$. Find $\nabla f(x)$ using the results from 
the previous exercises. What will the answer be if $\mathbf{H}$ is symmetric?
\end{problem}

\begin{problem}{4: A common case: The Lagrangian}
Given
\[
\mathcal{L}(x, \lambda, \mu) = xtr^{\top}\mathbf{G} + \lambda^{\top}(Cx - d) + \mu^{\top}(Ex - h)
.\] 

(a) Find $\nabla_x\mathcal{L}(x, \lambda, \mu)$.

\medskip

(b) Find $\nabla_{\mu}\mathcal{L}(x, \lambda, \mu)$.

\medskip

(c) Find $\nabla_{\lambda}\mathcal{L}(x, \lambda, \mu)$.
\end{problem}




\end{document}
