%%% QUESTION %%%

\begin{problem}{2: KKT example 2}
Consider
\[
  \min 2x_1 + x_2 \qquad s.t. \qquad x_1^2 + x_2^2 - 2 = 0
  .\]

(a) Find all points which satisfy the KKT conditions

\medskip

(b) Illustrate the gradients of the active constraint and the objective function at the KKT points.

\medskip

(c) What is the value of the Lagrange multiplier? Is this consistent with the KKT conditions?

\medskip

(d) Explain why each KKT point is (or is not) a local solution to the optimization problem.

\medskip

(e) Check the 2nd order conditions for the KKT points.

\medskip

(f) Is this problem a convex problem?

\end{problem}


%%% SOLUTION %%%

\SUBTASK{a}{KKT points}

The KKT points must satisfy $\nabla _x \mathcal{L}(\mathbf{x}, \lambda) = 0$.

\begin{align*}
  \mathcal{L}(\mathbf{x}, \lambda)           & = 2x_1 + x_2 - \lambda(x_1^2 + x_2^2 - 2)                         \\
  \nabla _x \mathcal{L}(\mathbf{x}, \lambda) & =
  \begin{bmatrix}
    2 - 2\lambda x_1 \\
    1 - 2\lambda x_2
  \end{bmatrix} = 0                                                                                              \\
  \implies \lambda                           & = \frac{1}{x_1} = \frac{1}{2x_2}                                  \\
  \implies x_2                               & = \pm \sqrt{ \frac{2}{5}},\  x_1 = 2x_2 = \pm \sqrt{ \frac{8}{5}}
\end{align*}

The KKT points are $\mathbf{x^{(1)}} = \M{\sqrt{ \frac{8}{5}} & \sqrt{ \frac{2}{5}}}^{\top}$ and $\mathbf{x^{(2)}} = \M{-\sqrt{ \frac{8}{5}} & -\sqrt{ \frac{2}{5}}}^{\top}$

\SUBTASK{b}{Gradients}

The gradiants are

\begin{align*}
  \nabla _x f(\mathbf{x}) = \M{2                          \\ 1} &,\
  \nabla c_1(\mathbf{x}) = \M{2x_1                        \\ 2x_2} \\
  \nabla c_1(\mathbf{x^{(1)}}) = \M{\sqrt{ \frac{16}{5}}  \\ \sqrt{ \frac{4}{5}}} &,\
  \nabla c_1(\mathbf{x^{(2)}}) = \M{-\sqrt{ \frac{16}{5}} \\ -\sqrt{ \frac{4}{5}}}
\end{align*}

\begin{center}
  \definecolor{qqttcc}{rgb}{0.,0.2,0.8}
  \definecolor{uuuuuu}{rgb}{0.26666666666666666,0.26666666666666666,0.26666666666666666}
  \definecolor{ududff}{rgb}{0.30196078431372547,0.30196078431372547,1.}
  \begin{tikzpicture}[line cap=round,line join=round,>=triangle 45,x=2.0cm,y=2.0cm]
    \begin{axis}[
        x=2.0cm,y=2.0cm,
        axis lines=middle,
        xmin=-3.5,
        xmax=3.5,
        ymin=-1.8,
        ymax=2.0,
        xtick={-3.5,-3.0,...,3.5},
        ytick={-1.5,-1.0,...,2.0},]
      \clip(-3.5,-1.8) rectangle (3.5,2.);
      \draw [line width=2.pt,color=qqttcc] (0.,0.) circle (2.8284271247461903cm);
      \draw [->,line width=2.pt] (1.2649110640673518,0.6324555320336759) -- (3.264911064067352,1.632455532033676);
      \draw [->,line width=2.pt] (1.2649110640673518,0.6324555320336759) -- (3.0537654460671835,1.5268827230335917);
      \draw [->,line width=2.pt] (-1.2649110640673518,-0.6324555320336759) -- (0.7350889359326482,0.3675444679663241);
      \draw [->,line width=2.pt] (-1.2649110640673518,-0.6324555320336759) -- (-3.0537654460671835,-1.5268827230335917);
      \begin{scriptsize}
        \draw [fill=ududff] (1.2649110640673518,0.6324555320336759) circle (2.5pt);
        \draw[color=ududff] (1.3306321521581599,0.8012016339412135) node {$\mathbf{x^{(1)}}$};
        \draw [fill=uuuuuu] (-1.2649110640673518,-0.6324555320336759) circle (2.0pt);
        \draw[color=uuuuuu] (-1.1988993296780108,-0.4774621023352682) node {$\mathbf{x^{(2)}}$};
        \draw[color=black] (2.377654340610494,1.3756737473407923) node {$\nabla$f};
        \draw[color=black] (2.136746580435621,1.255219917111848) node {$\nabla c(\mathbf{x^{(1)}})$};
        \draw[color=black] (-0.6429583446590723,-0.5423218570739303) node {$\nabla$f};
        \draw[color=black] (-2.0884049057083125,-1.0334028572380864) node {$\nabla c(\mathbf{x^{(2)}})$};
      \end{scriptsize}
    \end{axis}
  \end{tikzpicture}
\end{center}

\SUBTASK{c}{Lagrange multipliers}

We see that $\lambda^{(1)} = \frac{1}{x_1} = \frac{1}{\sqrt{ \frac{8}{5}}} = 0.79$ and $\lambda^{(2)} = -0.79$. The sign of the lagrange multiplier does however not 
matter as much in this problem because of the equality constraint. If we solved the equivalent problem with $c_1^{'}(\mathbf{x}) = -c_1(\mathbf{x})$, the lagrange multiplier
would have changed sign for the two points. 

\SUBTASK{d}{The solution}

$\mathbf{x^{(1)}}$ maximises f. We can see this from the illustration, where the constraint gradient points in the same direction as the objective function gradient. 
$\mathbf{x^{(2)}}$ is a local solution to the problem, since the constraint gradient is opposite to the objective gradient.

\SUBTASK{e}{2nd order conditions}

For this we need the Hessian. $\nabla _{xx}^2\mathcal{L}(\mathbf{x}, \lambda) = \M{-2\lambda & 0 \\ 0 & 2\lambda}$. 
We see that $\nabla _{xx}^2\mathcal{L}(\mathbf{x^{(1)}}, \lambda^{(1)}) = \M{-\sqrt{ \frac{16}{5}} & 0 \\ 0 & -\sqrt{ \frac{4}{5}}}$ is negative semi-definite. This means that 
$\mathbf{x^{(1)}}$ is a local maximum.
$\nabla _{xx}^2\mathcal{L}(\mathbf{x^{(2)}}, \lambda^{(2)}) = \M{\sqrt{ \frac{16}{5}} & 0 \\ 0 & \sqrt{ \frac{4}{5}}}$ is positive semi-definite, a local minimum.

Here, $d^{\top}\nabla _{xx}^2\mathcal{L}(\mathbf{x^{(2)}}, \lambda^{(2)})d > 0,\ d > 0$


\SUBTASK{f}{Convexity}

This is not a convex problem, because the feasible region is not convex. A line between any distinct points would leave the feasible region.