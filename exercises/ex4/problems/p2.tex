%%% QUESTION %%%

\begin{problem}{1: Production Planning and Quadratic Programming}
  Two reactors $R_1$ and $R_2$ produce two products A and B. To make 1000kg of A, 2 hours of $R_1$ and 1 hour of $R_2$ are required. 
  To make 1000kg of B, 1 hour of $R_1$ and 3 hours of $R_2$ are required. The order of $R_1$ and $R_2$ does not matter. $R_1$ and $R_2$ are available for 8 and 15 hours, respectively. 
  We want to maximize the profit from selling the two products. 

  The profit now depends on the production rate:
  \begin{itemize}
    \item the profit from A is $3-0.4x_1$ per tonne produced,
    \item the profit from B is $2-0.2x_2$ per tonne produced,
  \end{itemize}

  where $x_1$ is the production of product A and $x_2$ is the production of product B (both in number of tonnes). 
  
  \medskip (a) Formulate this as a quadratic program.
  
  \medskip (b) Make a contour plot and sktech constraints. 
  
  \medskip (c) Find the production of A and B that maximizes the total profit. 
  
  \medskip (d) The solution is calculated by an active-set method. Explain how the method works based on the sequence of iterations from (c). 
  
  \medskip (e) Compare the solution in (c) with Problem 2 (c) in Exercise 3 and comment. 

\end{problem}

%%% SOLUTION %%%

\[
  f(x) = -[(3-0.4x_1)x_1 + (2-0.2x_2)x_2] = -(3-0.4x_1)x_1-(2-0.2x_2)x_2 = \frac{1}{2}x^{\top}\M{0.8 & 0 \\ 0 & 0.4}x + \M{-3 & -2}x
.\] 


\begin{align*}
  \min_x \qquad& q(x) = \frac{1}{2}x^{\top}Gx+x^{\top}c \\ 
  \text{s.t.}\qquad & -2x_1-x_2 \geq -8 \\ 
  &-x_1-3x_2 \geq -15 \\ 
  &x_1\geq 0 \\ 
  &x_2 \geq 0
\end{align*}

\SUBTASK{b}{Contour}

\begin{center}
  \incfig{contour}{0.6}
\end{center}

\SUBTASK{c}{Solution}

Plugging the values from (a) into the script gave these outputs:

\begin{itemize}
  \item $\M{x_1^{(0)} & x_2^{(0)}}^{\top} = \M{0 & 0}^{\top}$
  \item $\M{x_1^{(1)} & x_2^{(1)}}^{\top} = \M{2.4 & 3.2}^{\top}$
  \item $\M{x_1^{(2)} & x_2^{(2)}}^{\top} = \M{2.25 & 3.5}^{\top}$
\end{itemize}

\begin{center}
  \incfig{contour_steps}{0.5}
\end{center}

The solution is not found at an intersection. Only one constraint is active.

\SUBTASK{d}{Active-set method}

Is an iterative approach to solving QP problems by choosing a subset of the constraints to be active. 
This gives us a reduced, equality constrained problem that we can solve. This solution can be checked for optimality, and if it is not optimal, add or remove constraints 
from the active set before repeating the process.  

\SUBTASK{e}{Compare with LP}

The solution in exercise 3 used the fact that an optimal solution had to lie in an intersection, and could therefore iterate over those. Here we cannot make that assumption
and must therefore use a completely different method for solving the problem. We see that the second and third points differ. 