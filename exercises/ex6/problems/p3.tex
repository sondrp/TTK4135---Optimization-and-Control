%%% QUESTION %%%

\begin{problem}{2: MPC and input blocking}

Plant model is given by 
\begin{align*}
  x_{t+1} &= \M{0 & 0 & 0 \\ 0 & 0 & 1 \\ 0.1 & -0.79 & 1.78}x_t + \M{1 \\ 0 \\ 0.1}u_t \\ 
  y_t &= \M{0 & 0 & 1}x_t
\end{align*}

where $y_t$ is a measurement. The process has been at the origin $x_t = 0$, $u_t = 0$ for a while, but at $t = -1$ a 
disturbance moved the process so that $x_0 = \M{0 & 0 & 1}^{\top}$. We wish to solve a finite horizon ($N < \infty$) optimal 
control problem with the cost (or objective) function 
\[
  f(y_1,\dots,y_N, u_0,\dots,u_{N-1}) = \sum_{t=0}^{N-1} \{y_{t+1}^2 + ru_t^2\} ,\ r > 0
.\] 

Use r = 1, N = 30. Input constraint:
\[
  -1 \leq u_t  \leq 1 ,\ t \in \M{0, N-1}
.\] 

Assume that full state information is available for control. 

\medskip (a) Solve the open loop problem.

\medskip (b) REduce the number of control variables $u_t$ by dividing the time horizon into 6 
block of equal length (5 time steps each) and require $u$ to be a constant on each of these blocks. 
We will do thi sby modifying $A_{eq}$ in equation
\[
  A_{eq}z = b_{eq}
\] 

which represents the system equations (27a) for all the time instants on the horizon. That is, the
right part of the matrix $A_{eq}$ will have fewer columns. Give the structure of the modified $A_{eq}$ and state how many columns the right part now has. 

\medskip (c) We now use a better input parametrization: we use the same number of blocks but the input blocks are now of increasing lengths: 1, 1, 2, 4, 8, and 14 time steps. 
Does the parametrization change the number of iterations quadprog uses to find the solution?

\medskip (d) Solve MPC Problem 2 (b), Exercise 5 and solve with this plant model.

\medskip (e) Modify your code to solve the same MPC problem as above, but with the two input blocking 
schemes from problem (b) and (c).

\medskip (f) What is the effect of using input blocking with MPC? Why do we choose blocks of increasing length? 
Discuss based on the results obtained above. 


\end{problem}

%%% SOLUTION %%%
