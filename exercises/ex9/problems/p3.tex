%%% QUESTION %%%

\begin{problem}{3: Nelder-Mead}

  This problem will minimize the Rosenbrock function using the Nelder-Mead algorithm (a derivative-free method). 
  
  \medskip (a) The Nelder-Mead method needs n+1 starting points for an n-dimensional function. What are
  the conditions on the n+1 initial points? Give an example of an invalid set of starting points. Describe in geometric terms 
  the difference between valid and invalid sets of starting points for a function of two variables.
  
  \medskip (b) Study the code for Nelder Mead. Run the provided with different initial points. 

  \medskip (c) Use $x = \M{1.2 & 1.2}^{\top}$ as the initial point (this point is close to the optimum $x^* = \M{1 & 1}^{\top}$) and run the algorithm. 
  Then start the algorithm from $x = \M{-1.2 & 1}^{\top}$, which is a more difficult starting point. Study the output and the plots. Compare results with Problem 1. 
  
  \medskip (d) Show that if $f$ is a convex function, the shrinkage step in the Nelder-Mead simplex method will not increase the average value of the 
  function over the simplex vertices. 
\end{problem}

%%% SOLUTION %%%

\SUBTASK{a}{Starting points}

The Nelder-Mead method requires n+1 starting points for an n-dimensional function. These points must satisfy the following 
conditions:
\begin{itemize}
  \item They must form a non-degenerate simples: In n dimensions, an n-dimensional simples is a set of n+1 independent points. 
  This ensures that the simplex spans an n-dimensional space and can be properly contracted, expanded, or reflected during the optimization process. 
  \item No collinear (or coplanar in higher dimensions) points: If the points lie on the same line in 2D (or on the same plane in 3D), the algorithm can't explore the space 
  effectively. 
\end{itemize}

An example of invalid starting points for a function of two variables could be $(0,0)$, $(1,1)$, (2,2)

\medskip The geometric difference between valid and ivalid sets is that a valid set forms a shape with the maximal number of dimensions. 
An invalid set uses points that lie on the same line in 2D, or on the same plane in 3D.

\SUBTASK{b}{Running Nelder-Mead}

Here is an example from starting point $\M{0.34 & 0.66}^{\top}$.

\begin{minipage}[c]{0.5\textwidth}
  \centering
  \includegraphics[width=0.9\textwidth]{figures/3b1.png}
\end{minipage}
\hfill
\begin{minipage}[c]{0.5\textwidth}
  \centering
  \includegraphics[width=0.9\textwidth]{figures/3b2.png}
\end{minipage}

The solution is reached in a little over 60 iterations. 

\SUBTASK{c}{Comparison with previous methods}

The easy point $\M{1.2 & 1.2}^{\top}$:

\begin{minipage}[c]{0.5\textwidth}
  \centering
  \includegraphics[width=0.9\textwidth]{figures/3c1.png}
\end{minipage}
\hfill
\begin{minipage}[c]{0.5\textwidth}
  \centering
  \includegraphics[width=0.9\textwidth]{figures/3c2.png}
\end{minipage}

The hard point $\M{-1.2 & 1}^{\top}$:

\begin{minipage}[c]{0.5\textwidth}
  \centering
  \includegraphics[width=0.9\textwidth]{figures/3c3.png}
\end{minipage}
\hfill
\begin{minipage}[c]{0.5\textwidth}
  \centering
  \includegraphics[width=0.9\textwidth]{figures/3c4.png}
\end{minipage}

For some reason the last iteration from the difficult starting point just goes to (0,0). Other than that 
they are almost the same as the other methods in number of iterations at least. 
